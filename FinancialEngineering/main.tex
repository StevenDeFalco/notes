% A note-taking template by Steven DeFalco
% github.com/StevenDeFalco/notes

\documentclass{article}

% import note styles
\usepackage{../styles}

% Heading information
\title{FE530-WS: Introduction to Financial Engineering Notes}
\author{Steven DeFalco}
\date{Spring 2024}


\begin{document}


\maketitle
\tableofcontents
\newpage


\section{Lecture 1}

Financial engineering is a multidisciplinary field involving financial theory, methods of engineering, tools of mathematics, and the practice of programming. Serves a key role in the customer-driven derivatives business: modeling, programming, and risk managing financial products. Traditionally, financial engineering refers to the use of derivatives to manage risk and create customized financial instruments. 

\subsection{Derivatives}

Forwards, swaps, and options are the main building blocks of financial engineering. Such instruments can be used separatedly to hedge specific risks or be combined to form complex structures that meet client needs. Derivatives allow investors and institutions to break apart (segment) risks. Conversely, derivatives can be used to manage risks on a joint basis: having a top down approach where risks are managed altogether. \\

A \define{forward} (long) means that you are agreeing to buy a stock at some future time for some decided price: there is an \bold{obligation} to make thus purchase. An \define{option} means that you have the right to purchase at some decided price at some specific time in the future. A \define{put option} is the right to sell at a given price at some point in the future. \\ 

The financial engineers responsible for devising complex instruments do so to satisfy the risk-retrun appetites of their clients. The risk may come in the form of an unlikely but potentiall very severe future loss. The embedded risk is not fully understood by firms entering into complex derivative transactions, or it may be the case that these risks are not fully communicated to senior managers and other stakeholders. Factors such as adverse macroeconomic activity, increased competition, and evolving technologies can cause major losses for financial institutions. 


\section{Lecture 2}

A risk-free asset refers to a bank deposit or a bond issued by a government. A risky security will typically be some stock (or foreign exchange, commodity, or BTC). The price fo the risky asset at time $t$ is given by $S(t)$, 
\begin{itemize}
  \item $S(0)$ is known---check the last closing market price 
  \item $S(1)$ who knows what tomorrow brings?
\end{itemize}
In this regard, the \emph{rate of return} on the risky asset is given by $$R_S = \frac{S(1) - S(0)}{S_0}$$ Later on we will work with continuous models, and log-returns $$R_S \approx \textrm{ln}(\frac{S(1)}{S(0)}) = ln(1 + R_S )$$ 

\subsection{Risk-Free Asset}

The risk-free position can be described as the amount held in a bank account. As an alternative to keeping money in a bank, investors may choose to invest in bonds, especially Treasury bonds; all bonds are subjected to interest rate risk. Regardless, we will make the assumption of risk-free asset that yields a fixed return over time. The price of the risk-free asset is given by $B(t)$ at time $t$, and its return is constant which is given by $$R_F = \frac{B(1) - B(0)}{B(0)}$$ Clearly, since $B(0)$ is fixed and the value of the payment at time $B(1)$ is known with certainty, then the risk-free is constant $R_F$.  
\begin{itemize}
  \item The future stock price $S(t)$ for $t>0$ is a random variable with at least two different values 
  \item The future price $B(t)$ for $T \geq 0$ of the risk-free security is a known number. 
\end{itemize}

\begin{remark}
  All stock and bond prices are strictly positive.
\end{remark}

Consider a representative investor who holds $x$ and $y$ shares of the risky and risk-free assets respectively. The investor's wealth at time $t$ is given by $$V(t) = xS(t) + yB(t)$$ We will consider the change in wealth over discrete time period using $$\Delta V(t) = V(t + \Delta t) - V(t)$$ Hence, the return on the portfolio is given by $$R_V = w_S R_S + w_B R_F$$ with $w_S$ and $w_B$ denote the weight of the total intial wealth allocated to risk and risk-free assets, respectively. \\ 

The return on the portfolio is a linear combination of the return on each asset. The future value of the portfolio can be written as $R_V$ $$V(1) = V(0)(1+R_V)$$ This indicates to understand the behavior of the portfolio over the next period. 
\begin{itemize}
  \item \bold{Divisibility}---An investor may hold any number $x$ and $y$ of risky and risk-free shares whether integer or fractional, negative, positive or zero, such that $x,y \in R$; the same applies for weights such that $w_S , w_B \in R$ 
  \item \bold{Liquidity} denotes no bounds on $x$ and $y$, and, hence, the weights $w_S$ and $w_B$. One can buy and sell assets in arbitrary quantities. Lack of liquidity implis less market participants are willing to buy/sell.  
  \item \bold{Short selling}---if the position is positive (negative), then it is a long (short) position. Longing (shorting) the risk-fre asset is equivalent to lending (borrowing) money. 
\end{itemize}
Divisibility assumes that investors can purchase fractional shares. \\ 

The price $S(t)$ of a share of stock is a random variable taking only finitely many values for $t > 0$. However, in the real world the number of possible different prices is finite. 

\subsection{No-Arbitrage Principle}

\begin{definition}[Arbitrage]
  Theoretical definition denotes the situation in which one buys and sells an asset simulatneously to create a profit with 100\% probability. Practically, it denotes exploiting mispricing to create profit. 
\end{definition}

In an efficient market, we assume that there is no-arbitrage since market participants already arbitrage away any mis-pricing. The \define{no-arbitrage principle} indicates that there is no admissible portfolio with initial value $V(0) = 0$ suvh that $V(1) > 0$ with non-zero probability. 

\subsection{One-Step Binomial Model}

In general, the choice of stock and bond prices in a binomial model is constrained by the no-arbitrage principle. Suppose that the possible up and down stock prices at time 1 are $$S(1) = S^{u} \textrm{with probability} \pi$$ $$S(1) = S^{d} \textrm{with probability} 1 - \pi$$ where $S^d < S^u$ and $\pi \in (0,1)$ In the binomial tree price models, it followx that if $S(0) = B(0)$, then it must hold that $$S^d < A(1) < S^u$$ The above property denotes that if the prices of the two assets are equal today, then the risk-free rate should be somewhere between two levels $$d < R_F < u$$ Intuitively, there should be an upside for entering the risky asset. At the same time, there is a positive probability that the stock would underperform. \\ 

Suppose that $R_F < d < u$, then the risk free asset is dominated by the risk asset. The net from the transaction leads to an arbitrage profit of $d- R_f > 0$ violating the no-arbitrage principle. 
\begin{definition}[Risk-Neutral Probability]
  Risk-neutral valuation denotes that there is a probability $\pi \in (0,1)$ for which the expected return on the risk asset equals the risk-free rate $$R_F = u \pi + d (1- \pi) \rightarrow \pi = \frac{R_F - d}{u -d}$$
\end{definition}

\subsection{Risk and Return}

It is typical to value choices in terms of risk and reward. \define{Reward} denotes expected return. \define{Risk} denotes the possibility of not getting the return or simply something bad happens. \define{Risk aversion} is a preference or a sure outcome over a gamble with higher or equal expected value. Investing in risky assets induces risk into the portfolio. Hence, one expects a risk-premium to invest in the risky asset versus the risk-free asset. Given a choice between two portfolios with the same expected return, a risk-averse investor would prefer the one with a lower risk. If the risk levels were the same, the investor wouldopt for higher return. 

\subsection{Forward Contracts}

A \define{forward contract} is contractual agreement between two parties to buy/sell an asset in the future for a given price $F$ 
\begin{itemize}
  \item the seller is shorting (delivering) the asset 
  \item The buyer is longing (acquiring) the asset
\end{itemize}

The existence of forward contract on an asset implies that one can acquire the asset in the forward market in the future. At the same time, one can acquire the asset in the future by purchasing it today. Th two transactions are economically equivalent in the future; hence, their prices should reflect that. \\ 

To acquire the asset in forward market next period, it will cost us $F$ at $t=1$; however, we can invest in $F/(1 + R_F)$ in the risk-free asset at $t=0$ to accumulate $F$ capital by $F=1$. On the other hand, we can purchase the asset on the spot at $t=0$ for $S(0)$. Since both transactions are economically equivalent at $t=1$, their cost should be equal too, such that $$\frac{F}{1 + R_F} = S(0) \rightarrow F = S(0)(1 + R_F)$$

\subsection{Call and Put Options}

Unliek forward contracts, options give the right (not obligation) to purchase the asset in the future for a fixed price; such price is known as \define{strike/exercise} and denoted by $K$. Call (put) correspond to the right of purchasing (selling) the asset. 

\begin{remark}
  A major distinction between forward and options is the fact taht trading options requires paying a premium at $t=0$ to participate. The premium should reflect the \emph{fair} value of this options, which is $C(0)$
\end{remark}

We can price any derivative as long as we know its future cash flows and the risk-neutral measure of the risky-asset. 

\section{Risk-Free Assets}

There are two main questions regarding how money changes its value in time\dots
\begin{itemize}
  \item \bold{Prospective}: What is the present value of an amount fo be paid or received at a certain time in the future? 
  \item \bold{Retrospective}: What is the value of an amount invested or borrowed in the past?
\end{itemize}

The return on the investment commencing at time $s$ and ending at time $t$ is denoted by $R_F (s,t)$ $$R_F (s,t) = \frac{V(t) - V(s)}{V(s)}$$ 
\begin{remark}
  As a general rule, interest rates will always refer to a period of one year
\end{remark}

Simple interst is not a realistic description of the value of money in the longer term. In the majority of cases, the interst already earned can be reinvested to attract even more interest, produxing a higher return than that implied in the previous discussion. \\ 

If $m$ interest payments are made per annum, the time between two consecutive payments measured in years will be $1/m$. The first interest payment being due at time $1/m$. Each interest payment will increase the principal by a factor of $1 + r/m$. Given that the interest rate $r$ remains unchanges, after $t$ years the future value of an initial principal $P$ will become $$V(t) = (1 + \frac{r}{m})^{tm} P$$ where $(1 + \frac{r}{m})^{tm}$ denotes the growth factor. It is clear to see the relationship to continuous compounding. The result can be established by setting the limit of $m \rightarrow \infty$, such that $$\textrm{lim}_{m \rightarrow \infty}V(t) = P \times \textrm{lim}_{m \rightarrow \infty} [(1 + \frac{r}{m})^{m}]^{t} \rightarrow Pe^{rt}$$ An \define{annuity} is a sequence of finitely many payments of a fixed amount due at equal time interval. Suppose that payments of an amount $C$ are to be made once a year for $n$ years, the first one due a year hence. \define{Perpetuity} is an infinite sequence of payments of a fixed amount $C$ occurring at the end of each year. 

\subsection{The Money Market}

The \define{money market} refers to tradign in very short-term debt investments, involvest large-volume trades between institutions and traders (wholesale leve), includes money market mutual funds bought by individual investors and money market accounts opened by bank customers (retail level). An example is a \bold{bond} which is a financial security promising the holder a sequence of guaranteed future payments. \\ 

\begin{definition}[Zero-Coupon Bonds]
  The simples cast of a bond is a zero-coupon bond, which involves just a single payment. The issuing institution promises to exchange the bond for a certain amount of money $F$ (known face value). The face value is paid on a give time $T$, called the \emph{maturity} date. 
\end{definition}
Zero coupon bonds may be long or short-term investments. The bonds can be held until maturity or sold on secondary bond markets. Short-term zero coupon bonds generally have maturities of less than one year and are called \define{bills}. The price of abond is given by $$V(t) = 100[\frac{1}{1+r}]^{T-t}$$ Bonds promising a sequence of payments are called \define{coupon bonds}. These payments consist of the face value due at maturity, and coupons paid regularly. 

\section{Portfolio Management}

An investment in a risky security always comes with the possibility of losses or poor performance. Risk by definition is something bad that might happen. Portfolio analysis provides guidance in dealing with risk. \\ 

Uncertainty denotes scatter of returns around certain level (reference). Natural candidate for the reference level is the expected value. The extent of scatter can be measured by the standard deviation. \define{Standard deviation} measures the distance between possible values and expetation. By definition, the stock standard deviation (\define{volatility}) is $\sigma_S = \sqrt{\mathbb{V}[R_S]}$. Suppose that you are taking additional rismk using leverage. This indicates that you can multiply your return by $l > 1$ $$R_S (l) = l \times R_S$$ The volatility would increase accordingly $\sqrt{\mathbb{V}[R_S (l)]} = l \sigma_S$. 

\subsection{The Wisdom of Diversification}

If $x_i$ denotes the numbner of shares purchased of asset $i$ for $i = 1,2$, then the initial wealth is $$V(0) = x_1 S_1 (0) + x_2 S_2 (0)$$ In this regard, we have $$w_i = \frac{x_i S_i (0)}{V(0}$$ And in relative terms $$\frac{V(T)}{V(0)} = w_1 \frac{S_1 (T)}{S_1 (0)} + w_2 \frac{S_2 (T)}{S_2 (0)}$$ This result is a linear function of the individual returns $$\frac{V(T)}{V(0)} = w_1  R_{S,1} + w_2 R_{S,2} + w_1 + w_2$$ Since we have $w_1 + w_2 = 1$, we know that $R_V = w_1 R_{S, 1} + w_2 R_{S,2}$. \\ 

It is more convenient to represent the asset returns using matrix forms. Let $R \in \mathbb{R}^2$ denote the return on the two assets. The portfolio return is given by $R_V = w^{T}R$. Since portfolio is determined at time $t=0$, then $w$ is deterministic. The portfolio mean return is given by $$\mathbb{E}[R_V] = \mu v = w_1 \mu_1 + w_2 \mu_2$$ The portfolio variance is $$\mathbb{V}[R_V] = \sigma^2 v = w_{1} \sigma_{1}^{2} + w_{2} \sigma_{2}^{2} + 2 w_1 w_2 \sigma_{12}$$ The \define{Global Minimum Variance Portfolio} (GMVP) is the one that attrains minimum portfolio variance. The portfolio is determined by solving the following optimization problem $$\textrm{min}_w \sigma^{2}_{V} = w^{T} \Sigma w$$ subject to $w^{T} 1 = 1$. The optimization problem for the GMVP can be represented in the following manner $$\textrm{min}_{w, \lambda}L(w) = w^{T} \Sigma w + \lambda (1 - w^{T} 1)$$ Taking the derivative with respect to $w$ is $$\frac{\partial L(w)}{\partial w} = 2 \Sigma w - \lambda 1 = 0$$ Thus, the GMVP is given by $$w_{GMV} = \frac{\Sigma^{-1}1}{1^{T} \Sigma^{-1}}1$$ Note that the covariance matrix is symmetric and positive definite. \\ 

The \define{Two Fund Separation Theorem} indicates the efficient set of optimal MV portfolios can be determined by a convex combination of two MV efficient portfolios $w_1$ and $w_2$ $$w_{MV}(\alpha) = \alpha w_1 + (1 - \alpha) w_2$$ for $\alpha \in \mathbb{R}$. \\ 

Since the $w_{GMV}$ is the one associated with the lowest risk, we can think about $\alpha$ as the degree of risk aversion. For $\alpha \in (0,1)$, a higher (lower) value indicates higher risk aversion (tolerance). In the \define{Sharpe Portfolio}, we say that if the agent allocates funds away from the GMVP, then the choice should reflect risk-reward trade off. About all other choices on the MV efficient frontier, the agent would choose the one that yeilds the best risk-reward trade off. The sharpe portfolio can be solved to this solution $$w_{SR} = \frac{\Sigma^{-1} \mu}{1^{T} \Sigma^{-1} \mu}$$ For a sequence of $\alpha$ values, the MV efficient frontier is obtained by plotting $\mu v (\alpha)$ versus $\sigma_{V}^{2}(\alpha)$ For a sequence of $\alpha$ values, the MV efficient frontier is obtained by plotting $\mu v (\alpha)$ versus $\sigma_{V}^{2}(\alpha)$. \\

\section{Forward and Future Contracts}

A \define{forward contrac} is an agreement ot buy or sell an asset 
\begin{itemize}
  \item on a fixed date in the future, called the \define{delivery time}
  \item for a price specified in advance, called the \define{forward price}
\end{itemize}
The party to the contract who agrees to sell the asset is said to be taking a \bold{short forward position}. The other party, obliged to buy the asset at delivery, is said to have a \bold{long forwad position}. The principal reason for entering into a forward contract is to become independent of the unknown future price of a risky asset (heding uncertainty). \\ 

\begin{remark}
  Let us denote the time when the forward contract is exchanged/agreed by $0$, the delivery time by $T$, the forward price by $F(0,T)$, and the market price of the underlying asset by $S(T)$. No payment is made by either party at time 0, when the forward contract is exchanged. At delivery, the party with a long forward position will beenfit if $F(0,T) < S(T)$. On the other hand, the party holding a short forward position will suffer a loss $S(T) - F(O,T) < 0$. 
\end{remark}

\begin{remark}
  If the contract is initiated at time $t< T$ rather than $0$, then $F(t,T)$ denotes teh forward price. 
\end{remark}

Let us denote the value of longing (shorting) the contract as $V_L (t,T)$ ($V_S (t,T)$). We note that $$V_L (t,T) = -V_S (t,T)$$ The major principle is the price of the forward contract should be \emph{fair}. Fair means that if both parties were to participate in the contract, tehn the contract must not be biased against one party. \\ 

Note that the payoff from each contract is stochastic since $S(T)$ in unknown. However, the price of the contract at time $0$ is fixed $F(0,T)$. For the contract to break even, then $$\mathbb{E}[V_L(t,T)] = \mathbb{E}[V_S (t,T)]$$ The breaking even price states that $$\mathbb{E}[S(T) - F(0,T)] \rightarrow F(0,T) = \mathbb{E}[S(T)]$$ The forward price $F(0,T)$, hence, reflects the expected value of the underlying asset. \\ 

Let $r$ dnote the risk-free rate under continuous compounding and assume that it is constant throughout the period in question. The existence of the derivative contract indicates that we can buy the asset via two alternatives 
\begin{itemize}
  \item The first alternative is to borrow $S(0)$ dollars to buy the stock at time $0$ and keep it until time $T$. 
  \item The second alternative is entering a long forward position in stock with a delivery at time $T$ and forward price $F(0,T)$. 
\end{itemize}
Since both alternatives are economically equivlent, then the price of the forwad contract is $F(0,T) = S(0) e^{rT}$. \\ 

\subsection{Futures}

One of the two parties to a forward contract will be losing money. There is a risk of default by the party suffering a loss. Futures contracts are designed to eliminate credit risk. Just like a forward contract, a futures contract involves an underlying asset and a specified time of delivery. \\ 

As in the case of a forward contract, it costs nothing to initiate a \define{future}s position. The difference lies in the cash flow during the lifetime of the contract. A futures contract involves a random cash flow, known as \define{marking to market}. Namely, at each time step $n=1,2, \dots, N$ the holder of a long futures position will receive the amount of $$f(n,T) - f(n-1, T)$$ in case it is potiive. If negative, the holder has to pay the amount. \\ 

To ensure that the obligations involved in a futures position are fulfilled, certain practical regulations are enforced. Each investor entering into a futures contract has to pay a deposit, called the \define{initial margin}. In the case of a long futures position, the amount $f(n,T) - f(n-1, T)$ is added to the deposti of positive (or subtracted if negative) on a daily basis (the opposite holds true for a short position). Any excess that builds up above the initial margin can be withdrawn by the investor. If the deposit drops below a certain level, called the \define{maintenance margin}, the clearning house will issue a marign call: requesting the investor to make a payment and restore the deposit to the level of the initial margin. \\ 

To long a forward contract, we enter an agreement today at no cost. At the same time, short a fraction $\theta = e^{-r(T-t)} \in (0,1)$ of the futures. For simplicity suppose that $S(t) > S(0)$, the m2m requires you to pay $$\theta [f(t,T) - f(0,T)] > 0$$ To meet the requirement, you borrow $\theta [f(t,T) - f(0,T)]$ at the risk-free rate. Increase the short position to single unit at no cost. \\ 

A simply way to hedge an exposure to stock price variations is to enter a forward contract. However, a contract of this kind may not be readily available. Another possibility is to hedge using the futures market, which is well developed, liquid, and protected from the risk of default. Heding an asset, we are concerned with the downside. Shorting the futures contract is associated with m2m payments which are invested or borrowed. \\ 

The difference between teh spot and the futures price is called the \define{basis} $$b(t,T) = S(t) - f(f,T)$$ The basis converges to zero as $t \rightarrow T$ since $f(T,T) = S(T)$. According to the no-arbitrage futures price, it follows that $$b(t,T) = S(t) - S(t)e^{r(T-t)} = S(t)[1- e^{r(T-T)}]$$ For zero dividend yield, note the basis since $r>0$. \\ 

Futures are determined by supply and demand; no-arbitrage principles assure prices are efficient. In \define{contango}, investors pay more for the underlying in the futures, i.e., $f(t,T) > S(t)$ for $t<T$; this premium is related to the cost of carry. Contango is also known as fowardation. A market is in \emph{backwardation} when the futures price is below the spot price for a particular asset: $f(t,T) < S(t)$ for $t < T$. Backwardation signals that investors are expecting asset prices to fall over time. \\ 

One way to benefit from contango is through arbitrage strategies. As futures contracts near expiration, this type of arbitrage increases. The spot and futures price actually converge as expiration approaches due to arbitrage, and contango diminsihes. Nonetheless, the shift between contango and backwardation can result in severe losses. \\ 

Basis risk denotes the risk arising after performing heding. For futures, this arises due to 
\begin{itemize}
  \item difference between the underlying asset and contract 
  \item difference between asset delivery and expiration
\end{itemize}
The optimal hedge aims to minimize such risk. 

\end{document}
