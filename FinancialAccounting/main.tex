% A note-taking template by Steven DeFalco
% github.com/StevenDeFalco/notes

\documentclass{article}

% import note styles
\usepackage{../styles}

% Heading information
\title{ACC200: Principles of Financial Accounting Notes}
\author{Steven DeFalco}
\date{Spring 2024}


\begin{document}


\maketitle
\tableofcontents
\newpage


\section{Accounting and the Business Environment}

\subsection{Why is Accounting Important?}

\define{Accounting} is the information system that measures business activities, processes the information into reports, and communicates the results to decision makers. We can divide accounting into two major fields: financial accounting and managerial accounting. \define{Financial accounting} provides information for external decision makers, such as outside investors, lenders, customers, and the federal government. \define{Managerial accounting} focuses on information for internal decision makers, such as the company's managers and employees. \\ 

\define{Certified Public Accountants} (CPAs) are licensed professional accountants who serve the general public. CPAs work for public accounting firms, businesses, government entities, or educational institutions. \define{Certified Management Accountants} (CMAs) are certified professionals who specialize in accounting and financial management knolwedge. 

\subsection{What are the Organizations and Rules that Govern Accounting}

In the United States, the \define{Financial Accounting Standards Board (FASB)}, a privately funded organization, oversees the creation and governance of accounting standards. The FASB works with governmental regulatory agencies like the \define{Securities and Exchange Commission (SEC)}. The SEC is the U.S. governmental agency that oversees the U.S. financial markets. It also oversees those organizations that set standards (like the FASB).  \\ 

\subsubsection{Generally Accepted Accounting Principles}

The guidelines for accounting information are called \define{Generally Accepted Accounting Principles (GAAP)}. GAAP is the main U.S. accounting rule book. The primary objective of financial reporting is to provide information useful for making investment and lending decisions. To be useful, information must be relevant and have \define{faithful representation}. Information that is faithfully representative is complete, neutral, and free from material error. These basic accounting assumptions and principles are part of the foundation of for the financial reports that companies present. 

\begin{itemize}
  \item The most basic concept in accounting is that of the \define{economic entity assumption}. An economic entity is an orgnization that stands apart as a separate economic unit. An entity refers to one business, separate from its owners. A business can be organized as a$\dots$ 
    \begin{itemize}
      \item \define{Sole Proprietorship}: A business with a single owners 
      \item \define{Partnership}: A business with two or more owners and not organized as a corporation
      \item \define{Corporation}: A business organized under state law that is a separate legal entity 
      \item \define{Limited-Liability Company (LLC)}: A company in which each member is only liable for his or her own actions. 
    \end{itemize}
  \item The \define{cost principle} states taht acquired assets and services should be recorded at their actual cost and not fair value. The cost principle means we record a transaction at the amount shown on the receipt---the actual amout paid. Even though the purchaser may believe the price is a bargain, the item is recorded at the price actually paid and not at the \emph{expected} cost. The cost principle also holds that the accounting records should continue reporting the historical cost of an asset over its useful life instead of adjusting the cost annually to fair value. \define{Fair value} represents the price that would be received if the asset was sold. 
  \item The \define{Going concern assumption} assumes that the entitiy will remain in operation for the foreseeable future. Under the going concern assumption, accountants assume that the business will remain in operation long enough to use existing resources for their intended purpose. 
  \item Accountants assume that the dollar's purchasing power is stable. This is the basis of the \define{monetary unit assumption}, which requires that the items on the financial statements be measured in terms of monetary unit. 
\end{itemize}

To handle conflicts of interest and to provide reliable information, the SEC requires publicly held companies to have their financial statemetns audited by independent accountants. An \define{audit} is an examination of a company's financial statements and reords. The independent accountants then issue an opinion that staes whether the financial statements give a fair picture of the company's financial situation. \\ 

The \define{Sarbanes-Oxley Act (SOX)} requires management to review internal control and take responsibility for the accuracy and completeness of their financial reports. In addition, SOX made it a criminal offense to falsify financial statements. SOX also created a new watchdog agency, the \define{Public Company Accounting Oversight Board (PCAOB)}, to monitor the work of independent accountants who audit public companies. 

\subsection{What is the Accounting Equation?}

The basic tool of accounting is the \define{accounting equation}. It measures the resources of a business (what the business owns or has control of) and the claims of those resources (what the business owes to creditors and to the owners). The accounting equation is made up of three parts---assets, liabilities, and equity---and shows how these three parts are releated. Assets appear on the left side of the equiation, and the liabilities and equity appear on the right side. $$\textrm{Assets = Liabilities + Equity}$$ An \define{asset} is an economic resource that is expected to benefit the business in the future. Assets are something of value that a business owns or has control of (e.g. cash, merchandise, inventory). Claims to those assets come from two sources: liabilities and equity. \define{Liabilities} are debts that are owed to creditors. Liabilities are something the business owes and represent the creditors' claims on the business' assets. The owners of a corporation are referred to as stockholders. The owners' claims to the assets of the business are called \define{equity}. Equity represents the amount of assets that are left over after the company has paid its liabilities. \\ 

Equity consists of two main components: contributed capital and retained earnings. Owner contributions to a corporation are referred to as \define{contributed capital}. A sotckholder can contribute cash or other assets to the business and receive capital. The basic element of contributed capital is stock, which the corporation issues to the stockholders as evidence of their ownership. \define{Common stock} represents the basic ownership of every corporation. \\ 

\define{Retained earnings} is the equity earned by profitable operations that is not distributed to stockholders. There are three types of events that affect retained earnings: dividends, revenues, and expenses. A profitable corporation may make distributes to stockholders in the form of \define{dividends}. Dividends can be paid in the form of cash, stock, or other property. \\ 

\define{Revenues} are earnings that result from delivering goods or services to customers. \define{Expenses} are the costs of selling goods or services. Expenses are the opposite of revenues and, therefore, decrease retained earnings and stockholders' equity. The difference between revenue and expenses is net income or net loss. \define{Net income} occurs when total revenues are greater than total expenses. A net loss is the opposite. A \define{net loss} occurs when total expenses are greater than total revenues. 

\subsection{How do You Analyze a Transaction?}

Accounting is based on actual transactions. A \define{transaction} is any event that affects the financial position of the business \emph{and} can be measured with faithful representation. Transactions affect what the company has (assets) , owes (liabilities), and/or its net worth (equity). An accountant only records events that have dollar amounts that can be measured reliably, such as the purchase of a building, a sale of merchandise, and the payment of rent. \\ 

\subsubsection{How do You Prepare Financial Statements?}

\define{Financial statements} are business documents that are used to communicate information needed to make business decisions. These statements are prepared in the order described below. 
\begin{itemize}
  \item An \define{income statement} provides information about profitability for a particular period for the company. Revenues - Expenses = Net Income or Net Loss 
  \item A \define{statement of retained earnings} informs users about how much of the earnings were kept and reinvested in the company. Beginning Retained Earnings + Net Income - Dividends for the period = Ending Retained Earnings 
  \item A \define{balance sheet} provides valuable information to financial statement users about economic resources the company has (assets) as well as debts the company owes (liabilities), and allows decision makers to determine their opinion about the financial position of the company. Assets = Liabilities + Stockholders' Equity 
  \item A \define{statement of cash flows} reports on a business' cash receipts and cash payments for a period of time.
\end{itemize}

The income statement presents a summary of a business entity's reventues and expenses for a period of time, such as a month, quarter, or year. The statement of retained earnings shows the changest in retained earnings for a business entity during a time period, such as a month, quarter, or year. The balance sheet lists a business entity's assets, liabilities, and stockholders' equity as of a specific date, usually the end of a month, quarter, or year. THe balance sheet is a snapshot of the entity. An investor or creditor can quickly assess the overall health of a business by viewing the balance sheet. The statement of cash flows reports the cash coming in and the cash going out during a period. If a transaction does not involve cash, such as the purchase of supplies on account, it will not be reported on the statement of cash flows. 

\subsection{How do You Use Financial Statements to Evaluate Business Performance?}

\define{Return on assets (ROA)} measures how profitably a company uses its assets. Return on assets is calculated by dividing the net income by average total assets. Average total assets is calculated by adding the beginning and ending total assets for the time period and then dividing by two. $$\textrm{Beginning total assets + Ending total assets / 2}$$

\section{Recording Business Transactions}

\subsection{What is an Account?}

The accoutning equation is made up of three parts or categories: assets, liabilities, and equity. Each category contains accounts. An \define{account} ist he detailed record of all increases and decreases that have occurrerd in an individual asset, liability, or equity during a specified period.

\subsubsection{Assets}

\define{Assets} are economic resources that are expected to benefit the business in the future---something the busienss owns or has control of that has value. Below is a list of asset accounts that most businesses use$\dots$ 
\begin{itemize}
  \item \bold{Cash}: a business' money. Includes bank balances, bills, coins, and checks. 
  \item \bold{Accounts Receivable}: A customer's promise to pay in the future for services or goods sold. Often described as \emph{on account}. 
  \item \bold{Notes Receivable}: A \emph{written} promise that a customer will pay a fixed amount of money (principlal) and \emph{interest} by a certain date in the future. Usually more formal than an accounts receivable. 
  \item \bold{Prepaid Expense}: A payment of an expense in advance. It is considered an asset because the prepayment provides a benefit in the future. Examples of prepaid expenses are prepaid rent, prepaid insurance, and supplies. 
  \item \bold{Land}: the cost of land a business uses in operations. 
  \item \bold{Building}: The cost of an office building, a store, or a warehouse. 
  \item \bold{Equipment, furniture, and fixtures}: the cost of equipment, furniture, and fixtures. A business has a separate asset account for each type. 
\end{itemize}

\subsubsection{Liabilities}

A \define{liability} is a debt---that is, something the business owes. A business generally has fewer liability accounts than asset accounts. Below is a list of common liability accounts$\dots$
\begin{itemize}
  \item \bold{Accounts Payable}: A promise made by the business to pay a debt in the future. Arises from a credit purchase. 
  \item \bold{Notes Payable}: A \emph{written} promise made by the business to pay a debte, usually involving \emph{interest}, in the future. 
  \item \bold{Accrued Liability}: An amount owed by not paid. A specific type of payable such as taxes payable, rent payable, and salaries payable. 
  \item \bold{Unearned Revenue}: Occurs when a compnay receives cash from a customer but has not provided the product or service. The promise to provide services or deliver goods in the future. 
\end{itemize}

\subsubsection{Equity}

The stockholders' claim to the assets of the business is called \emph{equity} or \emph{stockholders' equity}. Stockholders' equity is made up of contributed capital and earned capital. Contributed capiutal consists of common stock. Earned capital results from the earnings of delivering goods or services (revenues), the cost of selling goods or services (expenses), and the distributions of those earnings (dividends). Below are the separate accounts for each element of equity$\dots$ 
\begin{itemize}
  \item \bold{Common Stock}: Represents the net contributions of the stockholders in the business. Increases equity. 
  \item \bold{Dividends}: Distributions of cash or other assets to the stockholders. Decreases equity. 
  \item \bold{Revenues}: Earnings that result from delivering goods or services to customers. Increases equity. Examples include service revenue and rent revenue. 
  \item \bold{Expenses}: the cost of selling goods or services. Decreases equity. Examples include rent expense, salaries expense, and utility expense.
\end{itemize}

\subsubsection{Chart of Accounts}

A \define{chart of accounts} lists all company accounts along with the account numbers. Account numbers are just shorthand versions of the account names. 

\subsubsection{Ledger}

A \define{ledger} is a collection of all the accounts, the changes in those accounts, and their balances. A chart of accounts a ledger are similar in that they both list the account names and account numbers of the business. A ledger, though, provides more detail. It includes the increases and decreases of each account for a specific period and the balance of each account as a specific point in time. 

\subsection{What is Double-Entry Accounting?}

Accounting uses the double-entry system to record the dual effects of each transaction. \\ 

A shortened form of an account in the ledger is called the \define{T-account} because it takes the form of the capital letter T. The vertical line divides the account into its left and right sides, with the account name at the top. The left side of the T-account is called the \define{debit} side, and the right side is called the \define{credit} side. \\ 

Assets are always increased with a debit and decreased with a credit. Liabilities and equity are always increased with a credit and decreased with a debit. \\ 

All accounts have a normal balance. An account's \define{normal balance} appears on the side---either debit or credit---where we record an \emph{increase} in the accounts balance. An account with a normal debit balance may occasionally have a credit balance. That indicates a negative amount in the account. 

\subsection{How Do You Record Transactions?}

Accountants use source document to provide the evidence and data for recording transactions. Some source documents that businesses use include the following: 
\begin{itemize}
  \item \bold{Purchase invoices}. Documents that tell the business how much and when to pay a vendor for purchases on account, such as supplies. 
  \item \bold{Bank checks}. Documents that illustrate the amount and date of cash payments. 
  \item \bold{Sales invoices}. Documents provided to clients when a business sells services or goods; tells the business how much revenue to record. 
\end{itemize}
After accountants review the source documents, they are then ready to record the transactions. Transactions are first recorded in a \define{journal}, which is the record of transactions in date order. Journalizing a transaction records the data only in the journal---not in the ledger. The data must also be transferred to the ledger. THe process of transferring data from the journal to the ledger is called \define{posting}. We post from the journal to the ledger. Debits in the journal are posted as debits in the ledger and credits as credits---no exceptions. \\ 

The journalizing and posting process has five steps: 
\begin{enumerate}
  \item Identify the accounts and the accoutn type. 
  \item Decide whether each account increases or decreases, then apply the rules of debits and credits. 
  \item Record the transaction in the journal. 
  \item Post the journal entry to the ledger. 
  \item Determine whether the accounting equation is in balance. 
\end{enumerate}

\subsection{What is the Unadjusted Trial Balance?}

After the transactions are recorded in the journal and then posted to the ledger, a \define{trial balance} can be prepared. The trial balance summarizes the ledger by listing all the accounts with their balances---assets first, followed by liabilities, and then equity. The trial balance provides an accuracy check by showing whether total debits equal total credits. The trial balance is also a useful summary of the accounts and tehir balances because it shows the balances on a specific date for all acccounts in a company's accounting system. 

\subsubsection{Correcting Trial Balance Errors}

Balancing errors can be detected by computing the difference between total debits and total credits on teh trial balance. Then perform one or more of the following actions: 
\begin{enumerate}
  \item Search the trial balance for a missing account. 
  \item Divide the difference between total debits and total credits by 2. 
  \item Divide the out-of-balance amount by 9. 
\end{enumerate}
Total debits can equal total credits on the trial balance; however, there still could be errors in individual account balances because an incorrect account might have been selected in an individual journal entry. 

\subsection{What is the Accounting Cycle?}

The \define{accounting cycle} is the process by which companies produce their financial statements for a specific period of time. It is the steps that companies follow throughout the time period. Some of these steps are as follows: 
\begin{enumerate}
  \item Start with the beginning account balances 
  \item Analyze and journalize transactions in the journal 
  \item Post journal entries to the accounts in the ledger 
  \item Prepare the unadjusted trial balance 
\end{enumerate}

\subsection{How do You Use the Debt Ratio to Evaluate Business Performance?}

The \define{debt ratio} shows the proportion of assets financed with debt and is calculated by dividing total liabilities by total assets. It can be used to evaluate a business' ability to pay its debts. $$\textrm{Debt ratio} = \frac{\textrm{Total Liabilities}}{\textrm{Total Assets}}$$

\section{The Adjusting Process}

\subsection{Cash Basis Accounting vs. Accrual Basis Accounting}

\define{Cash basis accounting} records only transactions with cash: cash receipts and cash payments. When cash is received, revenues are recorded. When cash is paid, expenses are recorded. As a result, revenues are recorded only when cash is received and expenses are recorded only when cash is paid. The cash basis of accounting is not allowed under GAAP. \\ 

\define{Accrual basis accounting} follows GAAP and records the effect of each transaction as it occurs---that is, revenues are recorded when earned and expenses are recorded when incurred. Revenues are considered to be earned when the services or goods are provided to the customers. 

\subsection{What Concepts and Principles Apply to Accrual Basis Accounting?}

\subsubsection{The Time Period Concept}

Because businesses need periodic reports on their affairs, the \define{time period concept} assumes that a business' activities can be sliced into small time segments and that financial statements can be prepared for specific period, such as month, quarter, or year. The basic acounting period is one year, and most businesses prepare annual financial statements. The 12-month accounting period used for the annual financial statements is called a \define{fiscal year}. 

\subsubsection{The Revenue Recognition Principle}

The \define{revenue recognition principle} tells accountants when to record revenue and requires companies to follow a five-step process: 
\begin{enumerate}
  \item Identify the contract with the customer. 
  \item Identify performance obligations in the contract. 
  \item Determine the transaction price. 
  \item Allocate the transaction price to the performance obligations in the contract. 
  \item Recognize revenue when (or as) the entity satisfies each performance obligation. 
\end{enumerate}

\subsubsection{The Matching Principle}

The \define{matching principle} guides accounting for expenses and ensures the following: 
\begin{itemize}
  \item All expenses are recorded when they are incurred during the period. 
  \item Expenses are matched against the revenues of the period. 
\end{itemize}
To match expenses against revenues means to subtract expenses incurred during one month from revenues earned during that same month. The goal is to compute an accurate net income or net loss for the time period. 

\subsection{What are the Adjusting Entries for Deferrals?}

An \define{adjusting entry} is completed at the end of the accounting period and records revenues to the period in whcih they are earned and expenses to the period in which they occur. Adjusting entries also update the asset and liability accounts. Adjustments are needed to properly measure several items such as: 
\begin{enumerate}
  \item Net income (loss) on the income statement 
  \item Assets and liabilities on the balance sheet
\end{enumerate}
There are two basic categories of adjusting entries: \emph{deferrals} and \emph{accurals}. In a deferral adjustment, the cash payment occurs before an expense is incurred or the cash receipt occurs before the revenue is earned. A \define{deferral} delays (or defers) the recognition of revenue or expense to a date after the cash is received or paid. Accrual adjustments are the opposite. An \define{accrual} records an expense before the cash is paid, or it records the revenue before the cash is received. \\ 

The two basic categories of adjusting entries can be further separated into four types: 
\begin{enumerate}
  \item Deferred expense (deferral) 
  \item Deferred revenues (deferral) 
  \item Accured expenses (accrual) 
  \item Accrued revenues (accrual)
\end{enumerate}

\subsubsection{Deferred Expenses}

\define{Deferred expenses}, also called \emph{prepaid expenses}, are advance payments of future expenses. They are deferrals because the expense is not recognized at the time of payment but deferred until they are used up. Such payments are considered assets rather than expenses until they are used up. 

\subsubsection{Depreciation}

\define{Property, plant, and equipment} are long-lived, tangible assets used in the operation of a business. As a business uses these assets, their value and usefulness decline. The decline in usefulness of a plan asset is an expesne, and accountants ystematically spread the asset's cost over its useful life. The allocation of a plant's asset's cost over its useful life is called \define{depreciation}. \\ 

The expected value of a depreviable asset at the end of its useful life is called the \define{residual value}. The \define{straight-line method} for computing depreciation allocates an equal amount of depreciation each year and is calculated as $$\textrm{straight-line depreciation} = (\textrm{cost} - \textrm{residual value}) / \textrm{useful life}$$ The \define{accumulated depreciation} account is the sum of all depreciation expense recorded for the depreciable asset to date. Accumulated depreciation is a contra asset, which means that it is an asset account with a normal credit balance. Contra means opposite. A \define{contra account} has two main characteristics: 
\begin{itemize}
  \item a contra account is paired with and is listed immediately after its related account in the chart of accounts and associated financial statement 
  \item A contra account's normal balance (debit or credit) is the opposite of the normal balance of the related account. 
\end{itemize}
The net amount (cost minus accumulated depreciation) of a plant asset is called its \define{book value}. The book value represents the cost invested in the asset that the business has not yet expensed. 

\subsubsection{Deferred Revenues}

Deferred revenues occur when the company receives cash before it does the work or delivers a product to earn that cash. The company owes a product or a service to the customer, or it owes the customer his or her money back. Unearned revenue is a liability and is also called deferred revenue. 

\subsection{What are the Adjusting Entries for Accruals?}

\subsubsection{Accrued Expenses}

Businesses often incur expenses before paying for them. The term \define{accrued expense} refers to an expense of this type. An accrued expense hasn't been paid for yet. An accrued expense always creates an \define{accrued liability}. The formular for computing interest is as follows $$\textrm{Amount of interest} = \textrm{Principal} \times \textrm{Interest rate} \times \textrm{Time}$$ In the formula, time (period) represents the portion of a year that interest has accrued on the note. 

\subsubsection{Accrued Revenues}

Businesses can earn revenue before they receive the cash from their customers. This creates an \define{accrued revenue}, which is a revenue that has been earned but for which the cash has not yet been collected. 

\subsection{What is the Purpose of the Adjusted Trial Balance, and how do We Prepare It?}

After the adjustments have been journalized and posted, the account balances are updated, and an \define{adjusted trial balance} can be prepared by listing all the accounts with their adjusted balances. 

\subsection{Next Steps in the Accounting Cycle}

In this chapter, we discussed the next two steps of the accounting cycle. 
\begin{itemize}
  \item[Step 5] \bold{Journalize and post adjusting entries}. At the end of the accounting period, companies journalize and post adjusting entries to record revenues to the period in which they are earned and the expenses to the period in which they occur. 
  \item[Step 6] \bold{Prepare the adjusted trial balance}. An adjusted trial balance is prepared to summarize the account balances as reported in the ledger. 
\end{itemize}

\section{Completing the Accounting Cycle}

\subsection{How do We Prepare Financial Statements? (Classified Balance Sheets)}

Financial statements are prepared from the adjusted trial balance. Financial statements should always be prepared in the following order: income statement, statement of retained earning, balance sheet. We know that net income from the income statement increases retained earnings on the statement of retained earnings; a net loss decreases retained earnings. Then, the ending retained earnings from the statement of retained earnings goes to the balance sheet and makes total liabilities plus total stockholders' equity equal total assets. \\ 

In a \define{classified balance sheet}, each asset and liability are placed into a specific category or classification. Assets are show in order of liquidity and liabilities are classified by the order in which they must be paid, either \define{current} (within one year) or \define{long-term} (more than one year). \define{Liquidity} measures how quickly and easily an account can be converted to cash (because cash is the most liquid asset). \\ 

\subsubsection{Assets}

The balance sheet lists assets in the order of liquidity. A classified balance sheet reports two asset categories: \emph{current assets} and \emph{long-term assets}. \define{Current assets} will be converted to cash, sold, or used up during the next 12 months or within the business' operating cycle if the cycle is longer than a year. The \define{operating cycle} is the time span when: 
\begin{enumerate}
  \item Cash is used to acquire goods and services.
  \item These goods and services are sold to customers. 
  \item The business collects cash from customers. 
\end{enumerate}

For most businesses, the operating cycle is a few months. Cash, Accounts receivable, Supplies, and Prepaid Expenses are examples of current assets. \\ 

\define{Long-term assets} are all the assets that will not be converted to cash or used up within the business' operating cycle or one year, whichever is greater. Long-term assets are typically made up of three categories: long-term investments; propery, plant, and equipment; and intangible assets. \\ 

Notes receivable and other investments that are held long-term are considered \define{long-term investments} and include investments in bonds or stocks which the company intends to hold for longer than one year. Another category of long-term assets is \define{property, plant, and equipment}. Land, buildings, furniture, and equipment used in operations are plan assets. Assets with no physical form are \define{intangible assets}. Examples of intangible assets include patents, copyrights, and trademarks. 

\subsubsection{Liabilities}

The balance sheet lists liabilities in the order in which they must be paid. The two liability categories reported on the balance sheeet are \emph{current liabilities} and \emph{long-term liabilities}. \\ 

\define{Current liabilities} must be paid with cash, or with goods and services, within one year or within the entity's operating cycle if the cycle is longer than a year. Accounts payable, notes payable due within one year, salaries payable, interest payable, and unearned revenue are all current liabilities. Any portion of long-term liability that is due within the next year is also reported as a current liability. \\ 

\define{Long-term liabilities} are all liabilities that do not need to be paid within one year or within the entity's operating cycle, whichever is longer. 

\subsection{What is the Closing Process, and how do we Close the Accounts?}

The \define{closing process} consists of journalizing and posting the closing entries in order to get the accounts ready for the next period. The closing process zeroes out all revenue accounts and all expense accounts in order to measure each period's net income separately from all other periods. It also updates the retained earnings account balance for net income or loss during the period and any dividends paid to the stockholders. The closing process prepares the accounts for the next time period by setting the balances of revenues, expenses, and dividends to zero. \\ 

Revenues and expenses are called \define{temporary accounts}. The dividends account is also temporary and must be closed at the end of the period because it measures the payments to stockholders for only that one period. The balances of all temporary accounts do not carry forward into the next time period. Instead, the business starts the new time period with a zero beginning balance in termporary accounts. \\ 

By contrast, the \define{permanent accounts}---the assets, liabilities, common stock, and retained earnings---are not closed at the end of the period. Permanent account balances are carried forward into the next time period. All accounts on the balance sheet are permanent accounts. \\ 

\define{Closing entries} transfer the revenues, expenses, and dividends balances to the retained earnings account to prepare the company's books for the next period. As an intermediate step, the revenues and the expenses may be transferred first to an account titled Income Summary. The \define{income summary} account summarizes the net income (or net loss) for the period by collecting the sum of all the expenses (a debit) and the sum of all the revenues (a credit). 

After closing entries are recorde and posted, the accounting cycle ends with a \define{post-closing trial balance}. Only assets, liabilities, common stock, and retained earnings accounts appear on the post-closing trial balance. 

\subsection{The Current Ratio}

The \define{current ratio} measures a company's ability to pay its current liabilities with its current assets. This ratio is computed as follows: $$\textrm{Current ratio} = \textrm{Total current asset} / \textrm{Total current liabilities}$$ A company prefers to have a high current ratio because that means it has plenty of current assets to pay its current liabilities. A current ratio that has increased from the prior period indicates improvement in a company's ability to pays its current debts. A current ratio that has decreased from the prior period signals deterioration in the company's ability to pay its current liabilities. 

\begin{remark}
  A strong current ratio is 1.50, which indicates that the business has \$1.50 in current assets for every \$1.00 in liabiliites. A current ratio of 1.00 is considered low and somewhat risky. 
\end{remark}

\section{Merchandising Operations}

\subsection{What are Merchandising Operations?}

A \define{merchandiser} is a business that sells merchandies, or goods, to customers. The merchandise that this type of business sells is called \define{merchandise inventory}. Merchandisers are often identified as either wholesalers or retailers. A \define{wholesaler} is a merchandiser who buys goods from a manufacturer and then sells them to retailers. A \define{retailer} buys merchandise either from a manufacturer or w whilesaler and then sells those goods to customers. 

\subsubsection{The Operating Cycle of a Merchandising Business}

\begin{enumerate}
  \item It begins when the company purchases inventory from an individual or business, called a vendor 
  \item The company then sells the inventory to a customer 
  \item Finally, the company collects cash from customers 
\end{enumerate}

Because the operating cycle of a merchandiser is different than that of a service company, the financial statements differ. On the income statement, a merchandising company reports revenues using an account called \emph{Sales Revenue} rather than the account \emph{Service Revenue} used by service companies. A merchandiser also reports the cost of merchandise inventory that has been sold to customers, or \define{Cost of Goods Sold (COGS)}. Because COGS is usually a merchandiser's main expense, an intermediary calculation, gross profit, is determined before calculating net income. \define{Gross profit} is calculated as net sales revenue minus cost of goods sold and represents the markup on the merchandise inventroy. After calculating gross profit, operating expenses are then deducted to determine net income. \define{Operating expensses} are expenses, other than COGS, that occur in the entity's major ongoing operations. On the balance sheet, a merchandiser includes Merchandise Inventory in the current assets section representing the value of inventory that the business has on hand to sell to customers. 

\subsubsection{Merchandise Inventory Systems}

There are two main types of inventory accounting systems that are used: periodic inventory system and perpetual inventory system. \\ 

The \define{periodic inventory system} requires businesses to obtain a physical count of inventory to determine the quantities on hand. The system is normally used for relatively inexpensive goods, such as in small, local stores without optical scanning cash registers. \\ 

The \define{perpetual inventory system} keeps a running computerized record of merchandise inventory---that is, the number of inventory units and the dollar amounts associated with the inventory are perpetually (constantly) updated. 

\subsection{How are Purchases Recorded in a Perpetual Inventory System}

The \define{invoice} is the seller's request for payment from the buyer. An invoice is also called a \emph{bill}. \\ 

Sellers allow purchasers to return merchandise that is defective, damaged, or otherwise unsuitable. This is called a \define{purhcase return} from the purchaser's perspective. Alternatively, the seller may deduct an allowance from the amount the buyer owes. \define{Purchase allowances} are granted to the purchaser as an incentive to keep goods that are not \emph{as ordered}. Together, purchase returns and allowances decrease the buyer's cost of the merchandise inventory. \\ 

Many businesses offer purchasers a discount for early payment called a \define{purchase discount}. \define{Credit terms} express the discount, the discount period, and the final due date. A purchase discount is applied on the amount owed. If a business returns merchandise inventory or receives a purhcase allowance before payment is made, the purchase discount will be calculated net of the return or allowance. \\ 

Either the seller or the buyer must pay the transportation cost of shipping merchandise inventory. The purchase agreement specifies FOB (free on board) terms to determine when the title to the goods transfers to the purchaser and who pays the freight. 
\begin{itemize}
  \item \define{FOB shipping point} means the buyer takes ownership (title) to the goods when the goods leave the seller's place of business (shipping point). In most cases, the buyer also pays the freight. 
  \item \define{FOB destination} means the buyer takes ownership (title) to the goods at the delivery destination point. In most cases, the seller also pays the freight. 
\end{itemize}
When merchansiders are required to pay for shipping costs, those costs are classified as either freight in or freight out as follows: 
\begin{itemize}
  \item \define{Freight in} is the transportation cost to ship goods to the purchaser's warehouse; thus, it is freight on purchased goods 
  \item \define{Freight out} is the transportation cost to ship goods out of the seller's warehouse and to the customer; thus, it is freight on goods sold to a customer. 
\end{itemize}

The net cost of merchandise inventory purchase includes the purhcase cost of inventory, less purchase returns and allowances, less purchase discount, plus freight in. 

\subsection{How are Sales Recorded in a Perpetual Inventory System}

The amount a business earns from selling merchandise inventory is called \define{Sales Revenue} (sales). At the time of the sale, a company must record two entries in the perpetual inventory system: one entry records rthe sales revenue and the second entry records the cost of inventory sold (or COGS). The two journal entries are: 
\begin{itemize}
  \item A journal entry for the Sales Revenue and the Cash receieved 
  \item A journal entry for the expense (COGS) and the reduction of Merchandise Inventory
\end{itemize}

In a perpetual inventory system, the COGS account keeps a running balance throughout the period of the cost of merchandise inventory sold. 

\begin{remark}
  Credit card sales are recorded in the same manner as cash sales because the payment is usually received via an electronic transfer from the credit card processor within a few days. 
\end{remark}

\define{Sales discounts} decrease the amount of revenue earned on sales. Under the revenue recognition standards, sales can be recorded using either the gross or net method as long as revenue is reflected accurately on the income statement. \\ 

Allowances reduce the future cash collected from the customer. The return of goods or granting of an allowance is called \define{Sales Returns and Allowances}. Similar to Sales Discounts, the Sales Returns and Allowances account is a contra account to Sales Revenue and has a normal debit balance. The return of goods is called a \define{Sales Return}. Sales returns reduce the future cash collected from the customer or require a refund be made to the customer. 

\subsection{What are the Adjusting and Closing Entries for a Merchandiser}

A merchandiser ajusts and closes accounts in a similar manner that a service entity does. In addition, merchandisers must also adjust for inventory shrinkage. 

\subsubsection{Adjusting Merchandise Inventory for Shrinkage}

The merchandise inventory account should stay current at all times in a perpetual inventory system. However, the actual amount of inventory on hand may differ from what the books show. This difference can occur because of theft and damage and is referred to as \define{inventory shrinkage}. For this reason, businesses take a physical count of inventory \emph{at least} once a year. The most common time to count inventory is at the end of the fiscal year. The business then adjusts the Merchandise Inventory account based on the physical count.

\subsection{How are a Merchandiser's Financial Statements Prepared}

The financial statements for service businesses are also used by merchandiers. However, the merchandiser's financial statements will contain the new accounts that are unique to merchandisers. \\ 

\subsubsection{Net Sales Revenue and Gross Profit}

\define{Net Sales Revenue} is the first item listed on an income statement. It is calculated as Sales Revenue less Sales Returns and Allowances and Sales Discounts. Net Sales Revenue is the amount a company has earned on sales after sales returns and allowances and sales discounts have been taken out. 

\subsubsection{Multi-Step Income Statement}

The \define{single-step income statement} groups all revenues together and all expenses together without calculating other subtotals. A \define{multi-step income statement}, which is used by most merchandising companies, is different than a single-step income statement because it lists several important subtotals. In addition to net income, it also reports subtotals for gross profit, operating income, and income before income tax expense. \\ 

The income statement ebgins by calculating gross profit. Gross profit is the markup on the merchandise inventory and is calculated as net sales revenue minus cost of goods sold. Next, the operating expenses, those expenses other than cost of good sold that are related to the day-to-day operations of the business, are listed. Both merchandisers and service companies report operating expenses in two categories: 
\begin{itemize}
  \item \define{Selling expenses} are operating expenses related to marketing and selling the company's goods and servgices. These include sales salaries, sales commissions, advertising, depreciation on store buildings and equipment, store rent, utilities on store buildings, property taxes on store buildings, and delivery expense. 
  \item \define{Administrative expenses} include operating expenses \emph{not} related to marketing the company's goods and services. These include office expenses, such as the salaries of the executives and office employees; depreciation on office buildings and equipment; rent other than on stores; utilities other than on stores; and property taxes on the administrative office building. 
\end{itemize}

Gross profit minus operating expenses equals \define{operating income}. Operating income measures the resutls of the entity's major ongoing activities. \define{Other income and expenses} reports revenues and expenses that fall outside the business' main, day-to-day, regular operations. Corporations are required to pay income tax; therefore, the last section of the income statement is the \define{income tax expense} section. This section reports the federal and state income taxes that are incurred by the corporation. 

\subsection{Gross Profit Percentage}

The \define{gross profit percentage} measures the profitability of each sales dollar about the cost of goods sold and is computed as follows $$\textrm{Gross profit percentage} = \textrm{Gross profit} / \textrm{Net sales revenue}$$ The gross profit percentage reflects a business' ability to earn a profit on its merchandise inventory. The gross profit earned on merchandise inventory must be high enough to cover the remaining expenses and to earn net income. 

\begin{remark}
  A small increase in the gross profit percentage from last year to this year may signal an important rise in income. Conversely, a small decrease from last year to this year may signal trouble. 
\end{remark}

\end{document}
