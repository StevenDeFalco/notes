% A note-taking template by Steven DeFalco
% github.com/StevenDeFalco/notes

\documentclass{article}

% import note styles
\usepackage{../styles}

% Heading information
\title{ACC200: Principles of Financial Accounting Notes}
\author{Steven DeFalco}
\date{Spring 2024}


\begin{document}


\maketitle
\tableofcontents
\newpage


\section{Accounting and the Business Environment}

\subsection{Why is Accounting Important?}

\define{Accounting} is the information system that measures business activities, processes the information into reports, and communicates the results to decision makers. We can divide accounting into two major fields: financial accounting and managerial accounting. \define{Financial accounting} provides information for external decision makers, such as outside investors, lenders, customers, and the federal government. \define{Managerial accounting} focuses on information for internal decision makers, such as the company's managers and employees. \\ 

\define{Certified Public Accountants} (CPAs) are licensed professional accountants who serve the general public. CPAs work for public accounting firms, businesses, government entities, or educational institutions. \define{Certified Management Accountants} (CMAs) are certified professionals who specialize in accounting and financial management knolwedge. 

\subsection{What are the Organizations and Rules that Govern Accounting}

In the United States, the \define{Financial Accounting Standards Board (FASB)}, a privately funded organization, oversees the creation and governance of accounting standards. The FASB works with governmental regulatory agencies like the \define{Securities and Exchange Commission (SEC)}. The SEC is the U.S. governmental agency that oversees the U.S. financial markets. It also oversees those organizations that set standards (like the FASB).  \\ 

\subsubsection{Generally Accepted Accounting Principles}

The guidelines for accounting information are called \define{Generally Accepted Accounting Principles (GAAP)}. GAAP is the main U.S. accounting rule book. The primary objective of financial reporting is to provide information useful for making investment and lending decisions. To be useful, information must be relevant and have \define{faithful representation}. Information that is faithfully representative is complete, neutral, and free from material error. These basic accounting assumptions and principles are part of the foundation of for the financial reports that companies present. 

\begin{itemize}
  \item The most basic concept in accounting is that of the \define{economic entity assumption}. An economic entity is an orgnization that stands apart as a separate economic unit. An entity refers to one business, separate from its owners. A business can be organized as a$\dots$ 
    \begin{itemize}
      \item \define{Sole Proprietorship}: A business with a single owners 
      \item \define{Partnership}: A business with two or more owners and not organized as a corporation
      \item \define{Corporation}: A business organized under state law that is a separate legal entity 
      \item \define{Limited-Liability Company (LLC)}: A company in which each member is only liable for his or her own actions. 
    \end{itemize}
  \item The \define{cost principle} states taht acquired assets and services should be recorded at their actual cost and not fair value. The cost principle means we record a transaction at the amount shown on the receipt---the actual amout paid. Even though the purchaser may believe the price is a bargain, the item is recorded at the price actually paid and not at the \emph{expected} cost. The cost principle also holds that the accounting records should continue reporting the historical cost of an asset over its useful life instead of adjusting the cost annually to fair value. \define{Fair value} represents the price that would be received if the asset was sold. 
  \item The \define{Going concern assumption} assumes that the entitiy will remain in operation for the foreseeable future. Under the going concern assumption, accountants assume that the business will remain in operation long enough to use existing resources for their intended purpose. 
  \item Accountants assume that the dollar's purchasing power is stable. This is the basis of the \define{monetary unit assumption}, which requires that the items on the financial statements be measured in terms of monetary unit. 
\end{itemize}

To handle conflicts of interest and to provide reliable information, the SEC requires publicly held companies to have their financial statemetns audited by independent accountants. An \define{audit} is an examination of a company's financial statements and reords. The independent accountants then issue an opinion that staes whether the financial statements give a fair picture of the company's financial situation. \\ 

The \define{Sarbanes-Oxley Act (SOX)} requires management to review internal control and take responsibility for the accuracy and completeness of their financial reports. In addition, SOX made it a criminal offense to falsify financial statements. SOX also created a new watchdog agency, the \define{Public Company Accounting Oversight Board (PCAOB)}, to monitor the work of independent accountants who audit public companies. 

\subsection{What is the Accounting Equation?}

The basic tool of accounting is the \define{accounting equation}. It measures the resources of a business (what the business owns or has control of) and the claims of those resources (what the business owes to creditors and to the owners). The accounting equation is made up of three parts---assets, liabilities, and equity---and shows how these three parts are releated. Assets appear on the left side of the equiation, and the liabilities and equity appear on the right side. $$\textrm{Assets = Liabilities + Equity}$$ An \define{asset} is an economic resource that is expected to benefit the business in the future. Assets are something of value that a business owns or has control of (e.g. cash, merchandise, inventory). Claims to those assets come from two sources: liabilities and equity. \define{Liabilities} are debts that are owed to creditors. Liabilities are something the business owes and represent the creditors' claims on the business' assets. The owners of a corporation are referred to as stockholders. The owners' claims to the assets of the business are called \define{equity}. Equity represents the amount of assets that are left over after the company has paid its liabilities. \\ 

Equity consists of two main components: contributed capital and retained earnings. Owner contributions to a corporation are referred to as \define{contributed capital}. A sotckholder can contribute cash or other assets to the business and receive capital. The basic element of contributed capital is stock, which the corporation issues to the stockholders as evidence of their ownership. \define{Common stock} represents the basic ownership of every corporation. \\ 

\define{Retained earnings} is the equity earned by profitable operations that is not distributed to stockholders. There are three types of events that affect retained earnings: dividends, revenues, and expenses. A profitable corporation may make distributes to stockholders in the form of \define{dividends}. Dividends can be paid in the form of cash, stock, or other property. \\ 

\define{Revenues} are earnings that result from delivering goods or services to customers. \define{Expenses} are the costs of selling goods or services. Expenses are the opposite of revenues and, therefore, decrease retained earnings and stockholders' equity. The difference between revenue and expenses is net income or net loss. \define{Net income} occurs when total revenues are greater than total expenses. A net loss is the opposite. A \define{net loss} occurs when total expenses are greater than total revenues. 

\subsection{How do You Analyze a Transaction?}

Accounting is based on actual transactions. A \define{transaction} is any event that affects the financial position of the business \emph{and} can be measured with faithful representation. Transactions affect what the company has (assets) , owes (liabilities), and/or its net worth (equity). An accountant only records events that have dollar amounts that can be measured reliably, such as the purchase of a building, a sale of merchandise, and the payment of rent. \\ 

\subsubsection{How do You Prepare Financial Statements?}

\define{Financial statements} are business documents that are used to communicate information needed to make business decisions. These statements are prepared in the order described below. 
\begin{itemize}
  \item An \define{income statement} provides information about profitability for a particular period for the company. Revenues - Expenses = Net Income or Net Loss 
  \item A \define{statement of retained earnings} informs users about how much of the earnings were kept and reinvested in the company. Beginning Retained Earnings + Net Income - Dividends for the period = Ending Retained Earnings 
  \item A \define{balance sheet} provides valuable information to financial statement users about economic resources the company has (assets) as well as debts the company owes (liabilities), and allows decision makers to determine their opinion about the financial position of the company. Assets = Liabilities + Stockholders' Equity 
  \item A \define{statement of cash flows} reports on a business' cash receipts and cash payments for a period of time.
\end{itemize}

The income statement presents a summary of a business entity's reventues and expenses for a period of time, such as a month, quarter, or year. The statement of retained earnings shows the changest in retained earnings for a business entity during a time period, such as a month, quarter, or year. The balance sheet lists a business entity's assets, liabilities, and stockholders' equity as of a specific date, usually the end of a month, quarter, or year. THe balance sheet is a snapshot of the entity. An investor or creditor can quickly assess the overall health of a business by viewing the balance sheet. The statement of cash flows reports the cash coming in and the cash going out during a period. If a transaction does not involve cash, such as the purchase of supplies on account, it will not be reported on the statement of cash flows. 

\subsection{How do You Use Financial Statements to Evaluate Business Performance?}

\define{Return on assets (ROA)} measures how profitably a company uses its assets. Return on assets is calculated by dividing the net income by average total assets. Average total assets is calculated by adding the beginning and ending total assets for the time period and then dividing by two. $$\textrm{Beginning total assets + Ending total assets / 2}$$

\end{document}
