% A note-taking template by Steven DeFalco
% github.com/StevenDeFalco/notes

\documentclass{article}


% Packages...
\usepackage{listings} % to make nicely formatted code blocks
\usepackage{hyperref} % to get hot links in the table of contents
\usepackage{xcolor}   % to define custom colors
\usepackage{titlesec} % to make custom section formatting 

\newtheorem{example}{Example}[section]
\newtheorem{definition}{Definition}[section]
\newtheorem{theorem}{Theorem}[section]

% Custom Commands...

% \bold{} to bold the following text
% usage: \bold ==> \bold{bolded text}
\newcommand{\bold}[1]{\textbf{#1}}

% \define{} to bold and italicize text 
% usage: \define{<insert-definition>}
\newcommand{\define}[1]{\textbf{\textit{#1}}}

% \b to create a new list item
% usage: \b this is a bullet in my list
\renewcommand{\b}{\item[$\circ$]}

% \newlist to start an itemized list
% usage: \newlist \\ \b bullet 1 ...
\newcommand{\newlist}{\begin{itemize}}

% \listend to end an itemized list
% usage: ... \b {last bullet} \\ \endlist
\renewcommand{\endlist}{\end{itemize}}

% \code to format inline strings of code
% usage: \code{print("Hello world!")}
\newcommand{\code}[1]{\texttt{#1}}


% formatting defaults...

% removes paragraph indent by default
\setlength{\parindent}{0pt}

% custom colors for the code
\definecolor{codegreen}{rgb}{0,0.6,0}
\definecolor{light-gray}{gray}{0.95}
\definecolor{codegray}{rgb}{0.5,0.5,0.5}
\definecolor{codepurple}{rgb}{0.58,0,0.82}
\definecolor{backcolour}{rgb}{0.95,0.95,0.92}

% sets environment defaults for lstlistings (code blocks)
\lstset{
    backgroundcolor=\color{light-gray},   
    commentstyle=\color{codegreen},
    keywordstyle=\color{magenta},
    numberstyle=\tiny\color{codegray},
    stringstyle=\color{codepurple},
    basicstyle=\ttfamily\footnotesize,
    numbers=left,       
    numbersep=10pt,                  
    tabsize=2,
    frame=tb,
    stepnumber=1,
}

% sets custom paragraph environment
% layer of division that can be used with subsubsections
\setcounter{secnumdepth}{4}
\titleformat{\paragraph}
{\normalfont\normalsize\bfseries}{\theparagraph}{1em}{}
\titlespacing*{\paragraph}
{0pt}{3.25ex plus 1ex minus .2ex}{1.5ex plus .2ex}


% Heading information
\title{CS561: Database Management Systems Notes}
\author{Steven DeFalco}
\date{Fall 2023}


\begin{document}


\maketitle
\tableofcontents
\newpage


% Notes start here

\section{Introduction}

\define{Database management systems (DBMS)} consist of \bold{data}, \bold{software} (programs such as data interfaces), and \bold{environments} (operating systems). DBMS contains information about a particular enterprise. They are collections of interrelated data, a set of programs to access the data, and an environment that is both \emph{convenient} and \emph{efficient} to use. The \emph{user} should only have to define what it is that they want from the database, whereas the \emph{database} is responsible for defining how this query can be fulfilled; relational databases are good at this. \\

The three primary data models are \define{entity-relationship model} (diagrams), \define{relational model} (relational algebra), \define{relational database} (SQL). \\

Drawbacks to using \bold{file systems} to store data include the following: 

\begin{itemize}
  \item data \emph{redundancy} and \emph{inconsistency} (multiple formats, duplication of information, etc.)
  \item difficulties in \emph{accessing} data 
  \item data isolation (multiple files and formats) 
  \item concurrency issues (among multiple users)
  \item \emph{integrity} problems 
  \item atomicity of updates
  \item security problems (hard to provide varied levels of user access)
\end{itemize}

There are varying \define{levels of abstraction} in a database. The \define{physical level} defines how a record is stored. The \define{logical level} describes data stored in the database and the relationships among data. The \define{view level} is a way to hide details of data types and information for security purposes. \\ 

The \define{schema} is the logical structure of the database; this is analagous to type information of a variable in a program. \define{Physical schema} refer to database design at the physical level. \define{Logical schema} refer to database design at the logical level. An \define{instance} is the actual content of the database at a particular time; this is analagous to the value of a variable. \define{Physical data independence} is the ability to modify the physical schema without changing the logical schema. \\ 

\define{Data manipulation languages (DML)} are languages for accesing and manipulating the data organized in a DBMS. \define{Procedural languages} are ones in which the user specifies what data is required and how to get that data. \define{Declarative (nonprocedural) languages} are ones in which the user specifies what data is required without specifying how to get such data. \bold{SQL} is the most widely used query language. \\ 

A \define{data definition language (DDL)} is the specific notation for defining the database schema. The \define{DDL compiler} generates a set of tables stored in a data dictionary. Data dictionary contains metadata. \\ 

A \define{relational database} is based on the relational data model. Data and relationships among the data are represented by a collection of tables. These include both a \bold{DML} and \bold{DDL}. The most common relational database systems employ the \bold{SQL} query language. 

\section{Entity-Relationship Model}

A \emph{database} can be modeled as a collection of entities or a relationship among entities. An \define{entity} is an object that exists and is distinguishable from other objects (e.g. specific person, company, even, plant). These \emph{entities} have \define{attributes} (e.g. people hvae names and addresses). An \define{entity set} is a set of entities of the same type that share the same properties (e.g. set of all persons, companies). In the ER-model we refer to specific objects as \emph{entities} which have \emph{attributes} and are all a part of the entire \emph{entity set}. \\ 

A \define{relationship} is an association among several entties. A \define{relationship set} is a mathematical relation among $n \geq 2$ entties, each taken from entity sets. $$\{(e_1, e_2, \dots e_n) \vert e_1 \in E_1, e_2 \in E_2, \dots, e_n \in E_n\}$$ where $(e_1 , e_2 , \dots e_n)$ is a relationship. \\ 

\subsection{Attributes}

An \define{attribute} can also be property of a relationship set. For instance, the \emph{depositer} relationship set between entity sets \emph{customer} and \emph{account} may have the attribute \emph{access-date}. Relationshup sets that involve, two entity sets are \define{binary} (degree two). Relationship sets may involve more than two entity sets. Relationships between more than two entity sets are rare (i.e. most are binary). \\ 

An \define{entity} is represented by a set of attributes, that is descriptive properties possessed by all members of an entity set. \define{Domain} is the set of permitted values for each attribute. The \bold{types of attributes} include the following:

\begin{itemize}
  \item \emph{simple} (atomic) and \emph{composite} attributes 
  \item \emph{single-valued} and \emph{multi-valued} attributes
  \item \emph{derived} attributes (can be computed from other attributes)
\end{itemize}

When attributes are \emph{simple} and \emph{single-valued}, then we say that the data is in \define{First Normal Form}. 

\subsection{Keys}

A \define{super key} of an entity set is a set of one or more attributes whose values uniquely determine each entity. Once you have defined a \emph{super key}, you can add attributes and it is still considered a \emph{super key}. A \define{candidate key} of an entity set is a minimal super key (e.g. \emph{cusomer\_id} is a candidate key of \emph{customer}). Candidate keys \emph{only} contain the necessary attributes to make something unique. Although several candidate keys may exist, one of the candidate keys is selected to be the \define{primary key}. \\ 

The combination of primary keys of the participating entity sets forms a super key of relationship set. This means a pair of entity sets can have at most one relationship for each access. 

\subsection{E-R Diagrams}

Rectangles represent entity sets. Diamonds represent relationship sets. Lines link attributes to entity sets and entity sets to relationship sets. Ellipses represent attributes: double ellipses represent multivalued attributes while dashed ellipses denote derived attributes. Underline indicates primary key attributes. 

\section{Relational Model}

Formally, given sets $D_1 , D_2 , \dots D_n$ a \define{relation} $r$ is a subset of $$D_1 \times D_2 \times \dots \times D_n$$ Thus, a relation is a set of $n$-uples $(a_1 , a_2 , \dots a_n)$ where each $a_i \in D_i$. \\ 

Each attribute of a relation has a name. The set of allowed valued for each attribtue is called the \define{domain} of the attribute. Attribute values are (normally) required to be \define{atomic}; that is, indivisible. Domain is said to be atomic if all its members are atomic. The special value \emph{null} is a member of every domain. The null value causes complications in the definition of many operations. 
 
$A_1 , A_2 , \dots A_n$ are attributes. $R= (A_1 , A_2 , \dots , A_n)$ is a \define{relation schema}. $r(R)$ denotes a relation $r$ on the relation schema $R$. \\ 

The current values (\define{relation instance}) of a relation is specified by a table. An element $t$ of $r$ is a \emph{tuple}, represented by a row in a table. Relations are unordered and thus the roder of tuples is irrelevant (tuples may be stored in an arbitrary order). \\ 

A database consists of multiple relations. Information about an enterprise is broken up into parts, with each relation storing one part of the information. 

\end{document}
