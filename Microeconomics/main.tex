% A note-taking template by Steven DeFalco
% github.com/StevenDeFalco/notes

\documentclass{article}

% import note styles
\usepackage{../styles}

% Heading information
\title{BT244: Microeconomics Notes}
\author{Steven DeFalco}
\date{Spring 2024}


\begin{document}


\maketitle
\tableofcontents
\newpage


\section{Principles of Economics}

\define{Economics} is the study of how individuals make decision: how we produce, distribute, consume goods and services in a society where resources are scarce. The principles of economics are as follows: 
\begin{itemize}
  \item Choices are necessary because resources are scarce. 
    \begin{itemize}
      \item A \define{resource} is anything that can be used to produce something else. 
      \item A resource is \define{scarce} when there is not enough of the resource available to satisfy all the various ways a society wants to use it. 
    \end{itemize}
  \item The true cost of something is its \define{opporunity cost} (def: what you must give up in order to get something). 
  \item \emph{How much} is a decision at the margin (for each additional unit). 
    \begin{itemize}
      \item A \define{marginal decision} is a decision made at the margins of an acitvity about whether to do a bit more or a bit less of that activity 
      \item \define{Marginal analysis} is the study of margin decisions.
    \end{itemize}
  \item People respond to incentives (def: anything that offers rewards to people who change their behavior), exploiting opportunities to make themselves better off. 
\end{itemize}

\define{Trade} allows us to consume more than we otherwise could; gains from trade arise from specialization. \define{Specialization} is the situation in which each person specializes in the task that he or she is good at performing. The principles of the interaction of individual choices are as follows: 
\begin{enumerate}
  \item There are gains from trade. 
  \item Markets (def: a place where a good or service is exchanged) move toward equilibrium (def: an economic situation in which no individual would be better off doing something different)
  \item Resources should be used efficiently to achieve society's goals. Economy is efficient if it takes all opportunities to make some people better off without making other people worse off. 
  \item Markets usually lead to efficiency (def: all the opporunities to make people better off have been exploited), but when they don't, government intervention can improve society's welfare.
\end{enumerate}

\section{Economic Models: Trade-offs and Trade}

A \define{model} is a simplified representation of a real situation that is used to better understand real-life situations. By keeping our models simple, we can focus on the change of one variable, assuming everything else stays the same; this is the \emph{other things equal} assumption (ceteris paribus). 

\subsection{The Circular-Flow Diagram}

The \define{circular-flow diagram} represents the transactions in an economy by flows around a circle. 
\begin{definition}
  A \define{household} is a person or a group of poeple that share their income 
\end{definition}
\begin{definition}
  A \define{firm} is an organization that produces goods and services for sale. 
\end{definition}
Firms sells goods and services that they produce to households in \bold{markets for goods and services}. Firms buy the resources they need to produce goods and services in \bold{factor markets}. Main factors of production are land, labor, physical capital, and human capital. An economy's \bold{income distribution} is the way in which total income is divided among the owners of the various factors of production. 

\subsection{The Production Possibilities Frontier}

The \define{production possibilities frontier (PPF)} is a diagram that shows the combinations of two goods that are possible for a society to produce at full employment. \\ 

An economy is \bold{efficient} if there are no missed opportunities. An economy is \bold{efficient in production} if there are no missed opportunities in production. It is efficient in production if it could not produce more of any one good without producing less of something else---if it's on the PPF. An economy is inefficient in production if it could produce more of some things without producing less of others. The economy is \bold{efficient in allocation} if it allocates its resources so that consumers are as well of as possible. Efficiency requires both efficiency in production and efficiency in allocation. 

Economic growth means an expansion of the economy's production possibilities. Economic growth can be caused by: 
\begin{enumerate}
  \item \bold{An increase in factors of production}: resources used to produce goods and services (land, labor, physical capital, and human capital). 
  \item \bold{Better technology}: the technical means for producing goods and services. 
\end{enumerate}

\subsection{Positive versus Normative Economics}

\define{Positive economics} is the branch of economics analysis that \emph{describes} the way the economy actually works. \define{Normaltive economics} make \emph{prescriptions} about the way the economy should work. A \define{forecast} is a simple prediction of the future. 

\section{Supply and Demand}

Supply and demand analysis helps us understand factors impacting goods' prices and quantity in the economy. 

\subsection{Competitive Markets}

A \define{compteititve market} has \bold{many} buyers and sellers of the \bold{same} good or service, and none of whom can influence the price. Most markets are not perfectly competetive. The \define{supply and demand model} is a model of how a competetive market behaves. 

\subsection{Demand}

Demand represents the behavior of buyers. The \define{demand curve} is a plot that shows the quantity demanded of a good at each price level, assuming all other determinants of demand constant. A \define{demand schedule} is a table showing how much of a good or service consumers will want to buy at different prices. The \define{quantity demanded} is the quantity that buyers are willing (and able) to purchase at a particular price. The \define{law of demand} states that a higher price for goods leads people to demand smaller quantity of that good (other things equal). 

\begin{remark}
  Any change in price ($\Delta P$) will result in a movement along the demand curve (change in quantity demanded)
\end{remark}

\begin{remark}
  Any change in consumer income, consumer taste, or consumer preference will result in a movement of the entire demand curve (change in demand)
\end{remark}

A \bold{rightward} shift of the demand curve means an increase in demand. A \bold{leftward} shift of the demand curve means a decrease in demand. There are five factors that \bold{shift the demand curve}:
\begin{enumerate}
  \item Changes in the prices of related goods or services
  \item Changes in income 
  \item Changes in tastes 
  \item Changes in expectations 
  \item Changes in the number of consumers
\end{enumerate}

Two goods are \define{subtitutes} if a decrease in the price of one leads to a decrease in demand for the other (or vice versa); \emph{substitutes usually serve a similar function}. Two goods are \define{complements} if a decrease in the price of one good leads to an increase in the demand for the other (or vice versa); \emph{complements are usually consumed together}. \\ 

The effect of changes in income on demand depends on the nature of the good in question:
\begin{itemize}
  \item A \define{normal good}: demand increases when income increases. 
  \item An \define{inferior good}: demand decreases when income increases.
\end{itemize}

If consumers have a choice about the timing of a purchase, then they buy according to \bold{expectations}. Buyers adjust current spending in anticipation of the direction of future prices in order to obtain the lowest possible price. As the population of an economy changes, the number of buyers of a particular good also changes, thereby changing its demand. 

The \define{market demand curve} is the horizontal sum of the individual demand curves of all consumers. 

\subsection{Supply}

Supply represents the behavior of sellers. A \define{supply schedule} shows how much of a good or service would be supplied at different prices. A \define{supply curve} shows the quantity supplied at various prices. The \define{quantity supplied} is the quantity that producers are willing and able to sell at a particular price. \\ 

A \bold{rightward} shift of the supply curve means an increase in supply. A \bold{leftward} shift of the supply curve means a decrease in supply. Some factors that shif the supply curve are: 
\begin{enumerate}
  \item input prices 
  \item the prices of related goods or services 
  \item technology 
  \item expectations
  \item the number of producers
\end{enumerate}

An increase in the price of an input makes production more costly for sellers, and thus supply decreases. A fall in the price of an input makes production less costly for sellers, and thus supply increases. \\ 

Inputs used in production have \emph{opportunity costs}. Sellers will choose to use inputs whose profit is the higest\dots 
\begin{itemize}
  \item sellers will supply less of a good if its profitability falls 
  \item there are substitutes and complements in production processes 
\end{itemize}

New, better technology, enables producers to spend less on inputs, yet still produce the same amount of output: suplly increases. The expectation of a higher price for a good in the future decreases current supply of the good---if sellers can store the good. Sellers will adjust their current offerings in anticipation of the direction of future prices in order to obtain the highest possible price. As producers enter and exit the market, the overall supply changes:
\begin{itemize}
  \item \emph{Entry} implies more sellers in the market, increasing supply
  \item \emph{Exit} implies fewer sellers in the market, decreasing supply. 
\end{itemize}
On the supply side, firms have an incentive to charge the highest price possible. On the demand side, buyers have an incentive to search out the lowest price for the good. In a competetive market with many buyers and sellers, this will lead to one equilibrium price for the good. When $Q_S = Q_d$ at a certain price, the market is in \bold{equilibrium}. That is, the amount consumers would purchase at this price is matched exactly by the amount producers wish to sell. The price at which this takes place is the \define{equilibrium price}, also referred to as the market-clearing price. The quantity of the good or service bought and sold at that price is the \define{equilibrium quantity}. \\ 

There is a \define{surplus} of a good when the quantity supplied exceeds the quantity demanded. Surplusses occur when the price is above the equilibrium level. \bold{Surpluses} do not last: sellers will reduce price so they can move goods off the shelves. \\ 

There is a \define{shortage} when the quantity demanded exceeds the quantity supplied. Shortages occur when the price is below its equilibrium level. \bold{Shortages} do not last: sellers will realize they can charge higher prices. \\ 

An increase in demand leads to a movement along the supply curve to a higher equilibrium price and higher equilibrium quantity. An increase in supply leads to a movement along the demand curve to a lower equilibrium price and higher equilibrium quantity. \\ 

Simultaneous shifts in demand and supply can lead to ambigous changes to the quantity and price\dots 
\begin{itemize}
  \item \bold{Supply increases} and \bold{Demand increases}: Quantity increases, but price change is ambiguous 
  \item \bold{Supply increases} and \bold{Demand decreases}: Price decreases, but quantity change is ambiguous
  \item \bold{Supply decreases} and \bold{Demand increases}: Price increases, but quantity change is ambiguous 
  \item \bold{Supply decreases} and \bold{Demand decreases}: Quantity decreases, but price change is ambiguous
\end{itemize}

\section{Consumer and Producer Surplus}

The difference between your willingness to pay the price is \define{consumer surplus}. A consumer's willingness to pay for a good is the maximum price at whcih he or she would buy that good. \define{Individual consumer surplus} is the gain to an indvidual buyer from the purchase of a good; the difference between the price paid and what the buyer is willing to pay. \define{Total consumer surplus} is the sum of indivdual consumer surpluses of all buyers in a market. Economists often use the term \bold{consumer surplus} to refer to both individual and total consumer surplus.\\ 

\begin{remark}
  Consumer surplus is the area below the demand curve but above the price
\end{remark}

\define{Producer surplus} is the difference between market price and the price at which firms are willing to supply the product. \define{Individual producer surplus} is the net gain to an individual seller from selling a good. It is equal to the difference between the price received and the seller/s cost (the seller's cost includes monetary costs; it may also include other opportunity cost). \define{Total producer surplus} is the sum of individual producer surpluses of all the sellers in a market. Economists use the term \bold{producer surplus} to refer to both individual and total producer surplus. $$\textrm{Producer surplus} = \textrm{Price} - \textrm{Cost}$$ 

\begin{remark}
  Producer surplus is the area below the price, but above the supply curve. 
\end{remark}

\define{Total surplus} is the sum of the producer and consumers surpluses. Markets are usually efficient; there is no way to make some people better off without making other people worse off. Markets are usually efficient because they maximize total surplus. There are three ways which you might (unsuccessfully) try to increase the total surplus: 
\begin{enumerate}
  \item Reallocate consumption among consumers. 
  \item Reallocate sales among sellers. 
  \item Change the quantity traded.
\end{enumerate}

Competetive markets are usually efficient: 
\begin{enumerate}
  \item They allocate consumption of the good to the potential buyers who most value it.
  \item They allocate sales to the potential sellers who most value the right to sell the good (e.g. who have the lowest cost). 
  \item They ensure that all transactions are mutually beneficial. Every consumer who makes a purchase values the good more than every seller who makes a sale. 
  \item They ensure that no mutually beneficial transactions are missed. Every potential buyer who doesn't make a purchase values the good less than every potential seller who doesn't make a sale.
\end{enumerate}
There are, however, three caveats to efficiency:
\begin{enumerate}
  \item Although a market may be efficient, it isn't necessarily fair. 
  \item Markets sometimes fail to deliver efficiency. 
  \item Even when the market equilibrium maximizes total surplus, this doesn't mean that it results in the best outcomes for every individual consumer or producer. 
\end{enumerate}

\define{Property rights} are the rights of the owners of valuable items, whether resources or goods, to dispose of those items as they choose. Private property rights create and protect incentives to trade with others and to innovate. \\ 

An \define{economic signal} is any piece of information that helps people make better economic decisions. Prices are the most important signals in a market economy becuase they convey information about other people's costs and their willingness to pay. If prices are high, we can infer that demand for a given good is potentially high. 

\section{Price Controls and Quotas}

Market prices do not necessarily please buyers or sellers: they may lobby the government to help them by altering the price. \define{Price controls} are legal restrictions on how high or low a market price may go; there are two main types: 
\begin{itemize}
  \item A \define{price ceiling} is the maximum price sellers are allowed to charge for a good or service (e.g. rent control): usually set \emph{below} equilibrium 
    \begin{itemize}
      \item Side effects: inefficiently low quantity, inefficient allocation to customers, wasted resources, inefficiently low quality, black markets
      \item People expend money, effort, and time to cope with shortages caused by the price ceiling 
      \item A \define{black market} is a market in which goods or services are bought and sold illegally---either because they are prohibited or because the equilibrium price is illegal. Black markets encourage disrespect for the law and worsens the position of whose who are outside the market. 
    \end{itemize}
  \item A \define{price floor} is the minimum price buyers are required to pay for a good or service (e.g. minimum wage): usually set \emph{above} equilbrium
    \begin{itemize}
      \item Side effects: deadweight loss, inefficient allocation of sales among sellers, waste of resources, inefficiently high quality, temptation to break the law by selling below the legal price. 
      \item Price floors allow high-cost firms to operate and prevent low-cost firms from entering the industry. 
      \item Price flors encourage black markets because there are willing sellers (and buyers) at illegal prices, so they are tempted to break the law and trade with each other. 
    \end{itemize}
\end{itemize}

Sometimes governments control quantity instead of price. A \define{quota} is an upper limit, set by the government, on the quantity of some good that can be bought or sold; also referred to as a \define{quantity control}. A \define{quota limit} is the total amount of a good under a limit that can be legally transacted. A \define{license} is the right, conferred by the government, to supply a good. \\ 

The \define{demand price} is the price of a given quantity at which consumers will demand that quantity. The \define{supply price} is the price of a given quantity at which producers will supply that quantity. The \define{wedge, or quote, rent} is the difference between the demand price and the supply price at the quote limit: equal to the market price of the license when the license is traded. \\ 

Like price controls, quotas impose losses on society such as deadweight loss and black markets. 

\section{Elasticity}

\define{Price elasticity} of demand is the measure of price responsiveness. $$\textrm{Elasticity} = \frac{\Delta Q}{P\Delta P}$$ $$\emph{Price elasticity of demand} = \frac{\textrm{Percent change in quantity demanded}}{\textrm{Percent change in price}}$$ A demand is \define{elastic} when an increase in price reduces the quantity demanded a lot. A demand is \define{inelastic} when an increase in price reduces quantity demanded just a little. \\ 

The percent change calculation depends on the choice of starting point. To solve this problem, we calculate the price elasticity of demand using the midpoint formula for percentage changes. $$\textrm{Percent change in } x = \frac{\textrm{Change in } x}{\textrm{Average value of }x} \times 100$$ $$\textrm{Average value of } x = \frac{(\textrm{Starting value of } x + \textrm{Final value of } x)}{2}$$ A good can have a price elasticity as low as zero or as high as infinity. 
\begin{itemize}
  \item If a price elasticity $<1$, the demand curve is \bold{inelastic}. 
  \item If a price elasticity $>1$, the demand curve is \bold{elastic}
  \item If a price elasticity $=1$, the demand curve is \bold{unit-elastic}
\end{itemize}
\define{Total revenue} is equal to the price times the quantity sold $$\textrm{Total Revenue} = P \times Q$$ When demand is \bold{inelastic}, the price dominates the quantity effect, so an increase in price will cause only a slight reduction in the quantity demanded. In this instance, total revenue will rise when the price rises. When demand is \bold{elastic}, the quantity effect dominates the price effect, so an increase in price will cause a significant reduction in the quantity demanded. In this instance, total revenue will fall when price rises. When demand is unit-elastic, the quantity effect equals the price effect, so an increase in price exactly balances the reduction in the quantity demanded. In this instance, total revenue doesn't change. 

\subsection{Factors that Determine Price Elasticity of Demand}

Some factors that determine the price elasticity of demand are as follows\dots 
\begin{enumerate}
  \item Whether the good is a \bold{necessity} or a \bold{luxury}
    \begin{itemize}
      \item For \bold{necessities}, quantity demanded does not change much in response to a change in $P$. 
      \item For \bold{luxuries}, quantity demanded is more sensitive to a change in price. 
    \end{itemize}
  \item The \bold{availability of close substitutes}
    \begin{itemize}
      \item Fewer substitutes makes it harder for consumers to adjust $Q$ when $P$ changes, so \bold{demand is inelastic}. 
      \item Many substitues make it easier for consumers to switch brands when prices change, so \bold{demand is elastic}.
    \end{itemize}
  \item The \bold{sharte of income spent on the good} 
    \begin{itemize}
      \item It feels cheaper when we spend a smaller share of income on the good. 
      \item It feels more expensive when we spend a greater share of income on the good. 
    \end{itemize}
  \item \bold{Time elapsed since the price change}
    \begin{itemize}
      \item Less time to adjust means lower elasticity 
      \item \bold{Over time} consumers can adjust their behavior by finding substitues (making demand more elastic).
    \end{itemize}
\end{enumerate}

The \define{cross-price elasticity of demand} meausres how sensitive the quantity demanded of good A is to the price of good B. $$\textrm{Cross-price elasticity of demand} = \frac{\textrm{\% Change in quantity of A demanded}}{\textrm{\% Change in price of B}}$$ For \bold{substitutes}, cross-price elasticity of demand is \bold{positive}. For \bold{complements}, cross-price elasticity of demand is \bold{negative}.

The \define{income elasticity of demand} measures how sensitive the quantity demand of a good is to changes in income. $$\textrm{Income elasticity of demand} = \frac{\textrm{\% Change in quantity demanded}}{{\% Change in income}}$$ THe income elasticity of demand can be used to distinguish normal from inferior goods. For \bold{normal goods}, income elasticity is \bold{positive}. For \bold{inferior goods}, income elasticity is \bold{negative}. 

\begin{remark}
  For \bold{income-elastic goods}, income elasticity is greater than 1. For \bold{income-inelastic goods}, income elasticity is positive but less than 1. 
\end{remark}

Usually, sellers offer more when prices are higher, but how strong is that relationship? $$\textrm{Price elasticity of supply} = \frac{\textrm{\% Change in quantity supplied}}{{\% Change in price}}$$ The supply curve is \bold{elastic} if a rise in price increases the quantity supplied a lot. The supply curve is \bold{inelastic} if a rise in price increases the quantity supplied just a little. 

\section{Taxes}

Taxes drive a wedge between the price buyers pay and the price sellers receive. The \define{incidence} of a tax is a measure of who really pays it. The incidence of an excise tax doesn't depend on who officially pays the tax. 

\begin{remark}
  The most inelastic side of the market bears the majority of the tax burden.
\end{remark}

Increasing the tax rate does not necessarily increase revenue. On one side, the tax increase means that the government raises more revenue for each unit sold, which increases tax revenue. On the other side, it reduces the quantity of sales, which decrease tax revenue. 
\begin{itemize}
  \item If the price elasticities of both supply and demand are low, the tax increase won't reduce the quantity of the good sold very much, so tax revenue will definitely rise 
  \item If the price elasticities are high enough, the tax reduces the quantity sold so much that tax revenue falls 
  \item If the price elasticities are high, the result is less certain and dependent on how high or low the initial tax rate was
\end{itemize}
The \define{administrative costs} of a tax are the resources used for its collection, for the method of payment, and for any attempts to evade the tax. Taxes cost society time and effort that could have been used elsewhere. \\ 

There are two principles of tax fairness. The \define{benefits principle} states that those who benefit from public spending should bear the burden of the tax that pays for that spending. The \define{ability-to-pay principle} stats that those with greater ability to pay a tax should pay more. There is usually a trade-off between equity and efficiency: the system can be made more efficient only by making it less fair, and vice versa. \\ 

The \define{tax base} is the measure, such as income or property value, that determines how much tax an individual or firm pays. 
\begin{itemize}
  \item \define{Income tax} depends on income from wages and investments. 
  \item \define{Payroll tax} depends on the earnings an employer pays an employee.
  \item \define{Sales tax} depends on the value of goods sold.
  \item \define{Profits tax} depends on a firm's profit 
  \item \define{Property tax} depends on the value of property such as a home. 
  \item \define{Wealth tax} is a tax that depends on an individual's wealth. 
\end{itemize}
The \define{tax structure} specifies how the tax depends on the tax base. A \define{progressive tax} takes a larger share of the income of high-income taxpayers than of low-income taxpayers. A \define{regressive tax} takes a smaller share of the income of high-income taxpayers than that of low-income taxpayers. The \define{marginal tax rate} is the percentage of an increase in income that is taxed away. 

\section{Decision Making by Individuals and Firms}

Our decisions depend on comparing costs with benefits. Because resources are scarce, the true cost of anything is what you must give up to get it. An \define{explicit cost} is a cost that requires an outlay of money. An \define{implicit cost} does not require an outlay of money; it is measured by the value, in dollar terms, of benefits that are forgone. $$\textrm{Opportunity cost} = \textrm{Total explicit cost} + \textrm{Total implicit cost}$$ \define{Economic profit} equals revenue minus the opportunity cost of all resources used. \\ 

\define{Capital} is the total value of assets owned by an individual or firm---physical assets plus financial assets. The \define{implicity cost of capital} is the opportunity cost of the use of one's capital; that is, the income earned if the capital had been employed in its next best alternative use. \define{Sunk cost} is a cost that has already been incurrend and is not recoverable; a sunk cost should be ignored in decisions about future actions. \\ 

There are two types of decisions: a choice between two alternatives (\bold{either-or}) or a more complex choice that requires us to choose at the margin (\bold{how much}). When faced with an \bold{either-or} choice between two activities, choose the one with the positive economic profit. \bold{How-much} decisions are at the margin. \define{Marginal analysis} is the comparing the benefit of doing a little bit more of something with the cost of doing a little bit more of something. Must ask if marginal cost is greater than or less than marginal benefit. \define{Marginal cost} of producing a good or services is the additional cost incurred by producing one more unit of that good or service. \define{Marginal benefit} is the additional benefit derived from producing one more unit of a good or service. \\ 

The \define{marginal cost curve} shows how
 
\end{document}
