% A note-taking template by Steven DeFalco
% github.com/StevenDeFalco/notes

\documentclass{article}

% import note styles
\usepackage{../styles}

% Heading information
\title{BT244: Microeconomics Notes}
\author{Steven DeFalco}
\date{Spring 2024}


\begin{document}


\maketitle
\tableofcontents
\newpage


\section{Principles of Economics}

\define{Economics} is the study of how individuals make decision: how we produce, distribute, consume goods and services in a society where resources are scarce. The principles of economics are as follows: 
\begin{itemize}
  \item Choices are necessary because resources are scarce. 
    \begin{itemize}
      \item A \define{resource} is anything that can be used to produce something else. 
      \item A resource is \define{scarce} when there is not enough of the resource available to satisfy all the various ways a society wants to use it. 
    \end{itemize}
  \item The true cost of something is its \define{opporunity cost} (def: what you must give up in order to get something). 
  \item \emph{How much} is a decision at the margin (for each additional unit). 
    \begin{itemize}
      \item A \define{marginal decision} is a decision made at the margins of an acitvity about whether to do a bit more or a bit less of that activity 
      \item \define{Marginal analysis} is the study of margin decisions.
    \end{itemize}
  \item People respond to incentives (def: anything that offers rewards to people who change their behavior), exploiting opportunities to make themselves better off. 
\end{itemize}

\define{Trade} allows us to consume more than we otherwise could; gains from trade arise from specialization. \define{Specialization} is the situation in which each person specializes in the task that he or she is good at performing. The principles of the interaction of individual choices are as follows: 
\begin{enumerate}
  \item There are gains from trade. 
  \item Markets (def: a place where a good or service is exchanged) move toward equilibrium (def: an economic situation in which no individual would be better off doing something different)
  \item Resources should be used efficiently to achieve society's goals. Economy is efficient if it takes all opportunities to make some people better off without making other people worse off. 
  \item Markets usually lead to efficiency (def: all the opporunities to make people better off have been exploited), but when they don't, government intervention can improve society's welfare.
\end{enumerate}

\section{Economic Models: Trade-offs and Trade}

A \define{model} is a simplified representation of a real situation that is used to better understand real-life situations. By keeping our models simple, we can focus on the change of one variable, assuming everything else stays the same; this is the \emph{other things equal} assumption (ceteris paribus). 

\subsection{The Circular-Flow Diagram}

The \define{circular-flow diagram} represents the transactions in an economy by flows around a circle. 
\begin{definition}
  A \define{household} is a person or a group of poeple that share their income 
\end{definition}
\begin{definition}
  A \define{firm} is an organization that produces goods and services for sale. 
\end{definition}
Firms sells goods and services that they produce to households in \bold{markets for goods and services}. Firms buy the resources they need to produce goods and services in \bold{factor markets}. Main factors of production are land, labor, physical capital, and human capital. An economy's \bold{income distribution} is the way in which total income is divided among the owners of the various factors of production. 

\subsection{The Production Possibilities Frontier}

The \define{production possibilities frontier (PPF)} is a diagram that shows the combinations of two goods that are possible for a society to produce at full employment. \\ 

An economy is \bold{efficient} if there are no missed opportunities. An economy is \bold{efficient in production} if there are no missed opportunities in production. It is efficient in production if it could not produce more of any one good without producing less of something else---if it's on the PPF. An economy is inefficient in production if it could produce more of some things without producing less of others. The economy is \bold{efficient in allocation} if it allocates its resources so that consumers are as well of as possible. Efficiency requires both efficiency in production and efficiency in allocation. 

Economic growth means an expansion of the economy's production possibilities. Economic growth can be caused by: 
\begin{enumerate}
  \item \bold{An increase in factors of production}: resources used to produce goods and services (land, labor, physical capital, and human capital). 
  \item \bold{Better technology}: the technical means for producing goods and services. 
\end{enumerate}

\subsection{Positive versus Normative Economics}

\define{Positive economics} is the branch of economics analysis that \emph{describes} the way the economy actually works. \define{Normaltive economics} make \emph{prescriptions} about the way the economy should work. A \define{forecast} is a simple prediction of the future. 

\section{Supply and Demand}

Supply and demand analysis helps us understand factors impacting goods' prices and quantity in the economy. 

\subsection{Competitive Markets}

A \define{compteititve market} has \bold{many} buyers and sellers of the \bold{same} good or service, and none of whom can influence the price. Most markets are not perfectly competetive. The \define{supply and demand model} is a model of how a competetive market behaves. 

\subsection{Demand}

Demand represents the behavior of buyers. The \define{demand curve} is a plot that shows the quantity demanded of a good at each price level, assuming all other determinants of demand constant. A \define{demand schedule} is a table showing how much of a good or service consumers will want to buy at different prices. The \define{quantity demanded} is the quantity that buyers are willing (and able) to purchase at a particular price. The \define{law of demand} states that a higher price for goods leads people to demand smaller quantity of that good (other things equal). 

\begin{remark}
  Any change in price ($\Delta P$) will result in a movement along the demand curve (change in quantity demanded)
\end{remark}

\begin{remark}
  Any change in consumer income, consumer taste, or consumer preference will result in a movement of the entire demand curve (change in demand)
\end{remark}

A \bold{rightward} shift of the demand curve means an increase in demand. A \bold{leftward} shift of the demand curve means a decrease in demand. There are five factors that \bold{shift the demand curve}:
\begin{enumerate}
  \item Changes in the prices of related goods or services
  \item Changes in income 
  \item Changes in tastes 
  \item Changes in expectations 
  \item Changes in the number of consumers
\end{enumerate}

Two goods are \define{subtitutes} if a decrease in the price of one leads to a decrease in demand for the other (or vice versa); \emph{substitutes usually serve a similar function}. Two goods are \define{complements} if a decrease in the price of one good leads to an increase in the demand for the other (or vice versa); \emph{complements are usually consumed together}. \\ 

The effect of changes in income on demand depends on the nature of the good in question:
\begin{itemize}
  \item A \define{normal good}: demand increases when income increases. 
  \item An \define{inferior good}: demand decreases when income increases.
\end{itemize}

If consumers have a choice about the timing of a purchase, then they buy according to \bold{expectations}. Buyers adjust current spending in anticipation of the direction of future prices in order to obtain the lowest possible price. As the population of an economy changes, the number of buyers of a particular good also changes, thereby changing its demand. 

The \define{market demand curve} is the horizontal sum of the individual demand curves of all consumers. 

\subsection{Supply}

Supply represents the behavior of sellers. A \define{supply schedule} shows how much of a good or service would be supplied at different prices. A \define{supply curve} shows the quantity supplied at various prices. The \define{quantity supplied} is the quantity that producers are willing and able to sell at a particular price. \\ 

A \bold{rightward} shift of the supply curve means an increase in supply. A \bold{leftward} shift of the supply curve means a decrease in supply. Some factors that shif the supply curve are: 
\begin{enumerate}
  \item input prices 
  \item the prices of related goods or services 
  \item technology 
  \item expectations
  \item the number of producers
\end{enumerate}

An increase in the price of an input makes production more costly for sellers, and thus supply decreases. A fall in the price of an input makes production less costly for sellers, and thus supply increases. \\ 

Inputs used in production have \emph{opportunity costs}. Sellers will choose to use inputs whose profit is the higest\dots 
\begin{itemize}
  \item sellers will supply less of a good if its profitability falls 
  \item there are substitutes and complements in production processes 
\end{itemize}

New, better technology, enables producers to spend less on inputs, yet still produce the same amount of output: suplly increases. The expectation of a higher price for a good in the future decreases current supply of the good---if sellers can store the good. Sellers will adjust their current offerings in anticipation of the direction of future prices in order to obtain the highest possible price. As producers enter and exit the market, the overall supply changes:
\begin{itemize}
  \item \emph{Entry} implies more sellers in the market, increasing supply
  \item \emph{Exit} implies fewer sellers in the market, decreasing supply. 
\end{itemize}
On the supply side, firms have an incentive to charge the highest price possible. On the demand side, buyers have an incentive to search out the lowest price for the good. In a competetive market with many buyers and sellers, this will lead to one equilibrium price for the good. When $Q_S = Q_d$ at a certain price, the market is in \bold{equilibrium}. That is, the amount consumers would purchase at this price is matched exactly by the amount producers wish to sell. The price at which this takes place is the \define{equilibrium price}, also referred to as the market-clearing price. The quantity of the good or service bought and sold at that price is the \define{equilibrium quantity}. \\ 

There is a \define{surplus} of a good when the quantity supplied exceeds the quantity demanded. Surplusses occur when the price is above the equilibrium level. \bold{Surpluses} do not last: sellers will reduce price so they can move goods off the shelves. \\ 

There is a \define{shortage} when the quantity demanded exceeds the quantity supplied. Shortages occur when the price is below its equilibrium level. \bold{Shortages} do not last: sellers will realize they can charge higher prices. \\ 

An increase in demand leads to a movement along the supply curve to a higher equilibrium price and higher equilibrium quantity. An increase in supply leads to a movement along the demand curve to a lower equilibrium price and higher equilibrium quantity. \\ 

Simultaneous shifts in demand and supply can lead to ambigous changes to the quantity and price\dots 
\begin{itemize}
  \item \bold{Supply increases} and \bold{Demand increases}: Quantity increases, but price change is ambiguous 
  \item \bold{Supply increases} and \bold{Demand decreases}: Price decreases, but quantity change is ambiguous
  \item \bold{Supply decreases} and \bold{Demand increases}: Price increases, but quantity change is ambiguous 
  \item \bold{Supply decreases} and \bold{Demand decreases}: Quantity decreases, but price change is ambiguous
\end{itemize}

\section{Consumer and Producer Surplus}

The difference between your willingness to pay the price is \define{consumer surplus}. A consumer's willingness to pay for a good is the maximum price at whcih he or she would buy that good. \define{Individual consumer surplus} is the gain to an indvidual buyer from the purchase of a good; the difference between the price paid and what the buyer is willing to pay. \define{Total consumer surplus} is the sum of indivdual consumer surpluses of all buyers in a market. Economists often use the term \bold{consumer surplus} to refer to both individual and total consumer surplus.\\ 

\begin{remark}
  Consumer surplus is the area below the demand curve but above the price
\end{remark}

\define{Producer surplus} is the difference between market price and the price at which firms are willing to supply the product. \define{Individual producer surplus} is the net gain to an individual seller from selling a good. It is equal to the difference between the price received and the seller/s cost (the seller's cost includes monetary costs; it may also include other opportunity cost). \define{Total producer surplus} is the sum of individual producer surpluses of all the sellers in a market. Economists use the term \bold{producer surplus} to refer to both individual and total producer surplus. $$\textrm{Producer surplus} = \textrm{Price} - \textrm{Cost}$$ 

\begin{remark}
  Producer surplus is the area below the price, but above the supply curve. 
\end{remark}

\define{Total surplus} is the sum of the producer and consumers surpluses. Markets are usually efficient; there is no way to make some people better off without making other people worse off. Markets are usually efficient because they maximize total surplus. There are three ways which you might (unsuccessfully) try to increase the total surplus: 
\begin{enumerate}
  \item Reallocate consumption among consumers. 
  \item Reallocate sales among sellers. 
  \item Change the quantity traded.
\end{enumerate}

Competetive markets are usually efficient: 
\begin{enumerate}
  \item They allocate consumption of the good to the potential buyers who most value it.
  \item They allocate sales to the potential sellers who most value the right to sell the good (e.g. who have the lowest cost). 
  \item They ensure that all transactions are mutually beneficial. Every consumer who makes a purchase values the good more than every seller who makes a sale. 
  \item They ensure that no mutually beneficial transactions are missed. Every potential buyer who doesn't make a purchase values the good less than every potential seller who doesn't make a sale.
\end{enumerate}
There are, however, three caveats to efficiency:
\begin{enumerate}
  \item Although a market may be efficient, it isn't necessarily fair. 
  \item Markets sometimes fail to deliver efficiency. 
  \item Even when the market equilibrium maximizes total surplus, this doesn't mean that it results in the best outcomes for every individual consumer or producer. 
\end{enumerate}

\define{Property rights} are the rights of the owners of valuable items, whether resources or goods, to dispose of those items as they choose. Private property rights create and protect incentives to trade with others and to innovate. \\ 

An \define{economic signal} is any piece of information that helps people make better economic decisions. Prices are the most important signals in a market economy becuase they convey information about other people's costs and their willingness to pay. If prices are high, we can infer that demand for a given good is potentially high. 

\section{Price Controls and Quotas}

Market prices do not necessarily please buyers or sellers: they may lobby the government to help them by altering the price. \define{Price controls} are legal restrictions on how high or low a market price may go; there are two main types: 
\begin{itemize}
  \item A \define{price ceiling} is the maximum price sellers are allowed to charge for a good or service (e.g. rent control): usually set \emph{below} equilibrium 
    \begin{itemize}
      \item Side effects: inefficiently low quantity, inefficient allocation to customers, wasted resources, inefficiently low quality, black markets
      \item People expend money, effort, and time to cope with shortages caused by the price ceiling 
      \item A \define{black market} is a market in which goods or services are bought and sold illegally---either because they are prohibited or because the equilibrium price is illegal. Black markets encourage disrespect for the law and worsens the position of whose who are outside the market. 
    \end{itemize}
  \item A \define{price floor} is the minimum price buyers are required to pay for a good or service (e.g. minimum wage): usually set \emph{above} equilbrium
    \begin{itemize}
      \item Side effects: deadweight loss, inefficient allocation of sales among sellers, waste of resources, inefficiently high quality, temptation to break the law by selling below the legal price. 
      \item Price floors allow high-cost firms to operate and prevent low-cost firms from entering the industry. 
      \item Price flors encourage black markets because there are willing sellers (and buyers) at illegal prices, so they are tempted to break the law and trade with each other. 
    \end{itemize}
\end{itemize}

Sometimes governments control quantity instead of price. A \define{quota} is an upper limit, set by the government, on the quantity of some good that can be bought or sold; also referred to as a \define{quantity control}. A \define{quota limit} is the total amount of a good under a limit that can be legally transacted. A \define{license} is the right, conferred by the government, to supply a good. The \define{demand price} is the price of a given quantity at which consumers will demand that quantity. The \define{supply price} is the price of a given quantity at which producers will supply that quantity. 

\end{document}
