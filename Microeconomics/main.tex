% A note-taking template by Steven DeFalco
% github.com/StevenDeFalco/notes

\documentclass{article}

% import note styles
\usepackage{../styles}

% Heading information
\title{BT244: Microeconomics Notes}
\author{Steven DeFalco}
\date{Spring 2024}


\begin{document}


\maketitle
\tableofcontents
\newpage


\section{Principles of Economics}

\define{Economics} is the study of how individuals make decision: how we produce, distribute, consume goods and services in a society where resources are scarce. The principles of economics are as follows: 
\begin{itemize}
  \item Choices are necessary because resources are scarce. 
    \begin{itemize}
      \item A \define{resource} is anything that can be used to produce something else. 
      \item A resource is \define{scarce} when there is not enough of the resource available to satisfy all the various ways a society wants to use it. 
    \end{itemize}
  \item The true cost of something is its \define{opporunity cost} (def: what you must give up in order to get something). 
  \item \emph{How much} is a decision at the margin (for each additional unit). 
    \begin{itemize}
      \item A \define{marginal decision} is a decision made at the margins of an acitvity about whether to do a bit more or a bit less of that activity 
      \item \define{Marginal analysis} is the study of margin decisions.
    \end{itemize}
  \item People respond to incentives (def: anything that offers rewards to people who change their behavior), exploiting opportunities to make themselves better off. 
\end{itemize}

\define{Trade} allows us to consume more than we otherwise could; gains from trade arise from specialization. \define{Specialization} is the situation in which each person specializes in the task that he or she is good at performing. The principles of the interaction of individual choices are as follows: 
\begin{enumerate}
  \item There are gains from trade. 
  \item Markets (def: a place where a good or service is exchanged) move toward equilibrium (def: an economic situation in which no individual would be better off doing something different)
  \item Resources should be used efficiently to achieve society's goals. Economy is efficient if it takes all opportunities to make some people better off without making other people worse off. 
  \item Markets usually lead to efficiency (def: all the opporunities to make people better off have been exploited), but when they don't, government intervention can improve society's welfare.
\end{enumerate}

\section{Economic Models: Trade-offs and Trade}

A \define{model} is a simplified representation of a real situation that is used to better understand real-life situations. By keeping our models simple, we can focus on the change of one variable, assuming everything else stays the same; this is the \emph{other things equal} assumption (ceteris paribus). 

\subsection{The Circular-Flow Diagram}

The \define{circular-flow diagram} represents the transactions in an economy by flows around a circle. 
\begin{definition}
  A \define{household} is a person or a group of poeple that share their income 
\end{definition}
\begin{definition}
  A \define{firm} is an organization that produces goods and services for sale. 
\end{definition}
Firms sells goods and services that they produce to households in \bold{markets for goods and services}. Firms buy the resources they need to produce goods and services in \bold{factor markets}. Main factors of production are land, labor, physical capital, and human capital. An economy's \bold{income distribution} is the way in which total income is divided among the owners of the various factors of production. 

\end{document}
