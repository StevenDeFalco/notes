% A note-taking template by Steven DeFalco
% github.com/StevenDeFalco/notes

\documentclass{article}

% import note styles
\usepackage{../styles}

% Heading information
\title{BT243: Macroeconomics Notes}
\author{Steven DeFalco}
\date{Fall 2023}


\begin{document}


\maketitle
\tableofcontents
\newpage


% Notes start here

\section{Introduction}

\subsection{Basic Definitions}

\define{Scarcity} is a situation in which resources are limited and can be used in different ways, so we \emph{must sacrifice one thing for another}. \\ 

\define{Labor} is the primary source for providing goods and services. \\

\define{Opportunity cost} is what we give up when we make a choice or a decision. \\

\define{Economics} is the study of the choices made by people (\emph{individuals and societies}) who are faced with scarcity. The two major fields are microeconomics and macroeconomics. \emph{Microeconomics} studiees consumers and produces. \emph{Macroeconomics} studies the economy as a whole. \\ 

\subsubsection{Economic questions a society is faced with...}

\begin{enumerate}
  \item What will be produced? 
  \item How will it be produced?
  \item Who consumes the goods and services produced?
\end{enumerate}

\subsection{Production Possibility Frontier (PPF)}

The \define{PPF} is a graph that shows all possible combinations of goods and services that can be produced if all resources used efficiently. This is a limit that cannot be exceeded; it represents the best case scenario (in terms of efficiency). 

\begin{example}
Consider two goods: defense goods and non-defense goods. With limited resources only certain combinations can be produced; these include the following:
  \begin{itemize}
    \item[A] 200 units of defense goods and 0 units of non-defense goods
    \item[B] 195 units of defense goods and 25 units of non-defense goods (opp. cost = 5)
    \item[C] 188 units of defense goods and 50 units of non-defense goods (opp. cost = 7) 
    \item[D] 175 units of defense goods and 75 units of non-defense goods (opp. cost = 13)
    \item[E] 155 units of defense goods and 100 units of non-defense goods (opp. cost = 20)
    \item[F] 125 units of defense goods and 125 units of non-defense goods (opp. cost = 30)
    \item[G] 75 units of defense goods and 150 units of non-defense goods (opp.cost = 50)
    \item[H] 0 units of defense goods and 160 units of non-defense goods (opp. cost = 75)
  \end{itemize}
\emph{Opportunity cost increases} as specialization in inputs to the labor must be given up. 
\end{example}

Graphically, the \define{PPF} represents the barrier between inefficient use of resources and unachievable levels of prdouction given the scenario. A coordinate under the PPF curve indicates under-utilization of resources A coordinate above the PPF is an impossible combination and, by definiton, unachievable. \\

The PPF can be shifted (in the positive direction) if there is an \emph{increase in resurces} or an \emph{improvement in technology}; this is called \define{economic growth}. \define{International trade} can help a nation move beyond their maximum capabilities in terms of consumption (exceed their PPF). 

\section{Demand and Supply Model}

\define{Firms} produce and supply their output to the \define{consumers} who demand the product. 

\subsection{Demand}

The quantity (Q) \define{demanded} is the amount of a good or service that consumers are willing and able to buy. Willingness and ability are the primary determinants of demand. Price of a product is the main determinant of our willingness and ability to purchase a product. \\ 

\bold{Determinants} of demand: 
\begin{enumerate}
  \item Price of the product (P) \\ 
    \begin{itemize}
      \item As the price of a product increases, the quantity of the product demanded decreases. \\ 
      \item As the price of a product decreases, the quantity of the product demanded increases. \\ 
      \item The \define{Law of Demand} is the negative relationship between price and quantity. 
    \end{itemize}
  \item Income (M)
    \begin{itemize}
      \item As income increases, the quantity of products demanded (typcially) increases. 
      \item As income decreases, the quantity of products demanded decreases. 
      \item For some products these relationships are the opposite. These products are considered \define{inferior goods}; such products are demanded more when income is lower due to the nature of the product (e.g. fast food or bus rides). 
    \end{itemize}
  \item Prices of related goods 
    \begin{itemize}
      \item When an increase in the price of one good causes the demand for another good to increase, the two goods are called \define{substitutes}. 
      \item For example eating at restaurants and at-home can be considered \emph{substitutes}. If the price of eating at restaurants increases and causes a greater demand for eating-at-home/grocery-shopping (for example), then these two products are considered \bold{substitutes}. 
      \item When an increase in the price of one good causes the demand for another good to decrease, the two goods are called \define{complements}. 
      \item For example gasoline and big-cars can be considered \emph{complements} because as the price of gasoline increases, the demand for big-cars (that burn a lot of gas!) decreases. 
    \end{itemize}
  \item Taste and preferences of consumers 
  \item Expectations of consumers 
    \begin{itemize}
      \item This refers to consumers' beliefs about future income and prices.
    \end{itemize}
\end{enumerate} 

\begin{example}
  Ice-cream cones price-quantity relationship
  \begin{center}
    \begin{tabular}{ | c | c | }
      \hline
      \bold{P(\$)} & \bold{Q} \\
      \hline
      0 & 12 \\ 
      0.5 & 10 \\ 
      1 & 8 \\ 
      1.5 & 6 \\ 
      2 & 4 \\ 
      2.5 & 2 \\ 
      3 & 0 \\ 
      \hline 
    \end{tabular}
  \end{center}

  There is a \emph{negative relationship} between P and Q. The graph of price vs. quantity is the \bold{demand curve}. The \emph{demand curve} is linear. If income changes, the original relationship between price and quantity changes; thus, the demand curve will shift.

\end{example}

\begin{remark}
  A change in price results in movement along the demand curve. 
\end{remark}

\begin{remark}
  Demand curve shifting...
  \begin{itemize}
    \item The demand curve shifts to the right when there is an increase in demand. 
    \item The demand curve shifts to the left when there is a decrease in demand.  
  \end{itemize}
\end{remark}

\begin{definition}
  \define{celeris paribus} means "other things being equal". For example when modeling examples, we draw conclusions given that everything besides what we specifically study is constant. 
\end{definition}

\begin{definition}[Market demand]
  the sum of all individual demands. 
\end{definition}

\subsection{Supply}

The \define{quantity supplied} is the amount of a good or service that sellers are willing and able to sell. Some determinants of the \emph{quantity supplied} are as follows$\dots$

\begin{enumerate}
  \item \bold{Price (P)}: as the price (of the product/service) increases, the quantity supplied increases; as the price decreases, the quantity supplied decreases. This is called the \define{law of supply}. 
  \item \bold{Cost of production}: as the cost of production increases, the quantity supplied decreases; as the cost of production decreases, the quantity supplied increases. 
  \item \bold{Number of producers}
  \item \bold{Expectations of the firms}: beliefs about future performance
\end{enumerate}

The \define{supply curve} shows the positive relationship between price and quantity (P/Q). A change in the price results in movement along the supply curve. If any of the other determinants change then there is a shift of the supply curve. \\ 

\begin{remark}
  An increase in quantity supplied results in a \emph{right-shift} of the supply curve. A decrease in the quantity supplied results in a \emph{left-shift} of the supply curve. 
\end{remark}

\begin{definition}[Market supply]
  the sum of all that is supplied by all producers.
\end{definition}

\begin{definition}[Market equilibrium]
  a situation in which the quantity demanded is equal to the quantity supplied. $$Q_{D}=Q_{S}$$ The market equilibrium is---graphically---where the supply and demand curves intersect: when the quantity demanded and supplied are exactly the same. 
\end{definition}

Markets that are \emph{not} in equilibrium are \bold{inefficient}. 
\begin{itemize}
  \item \define{Excess supply (surplus)} refers to a scenario where producers are willing to produce more than consumers are willing to buy. In this case, the price will start to decrease and the quantity supplied decreases/quantity demanded increases until equilibrium is reached. 
  \item \define{Excess demand (shortage)} refers to a scenario where the consumers are willing to buy more than producers are willing to produce. The price will start to increase and---thus---the quantity demanded will decrease and the quantity produced will increase until equilibrium is met. 
\end{itemize}

\begin{example}{Solving for equilibrium}
  \\ Demand curve: $Q_D = 1600 - 300P$
  \\ Supply curve: $Q_S = 1400 + 700P$  \\ 
  Equilibrium: $Q_D = Q_S$ 
  $$1600-300P=1400+700P$$
  $$1000P=200$$
  $$P=0.2$$
  Substitute $P$ into either $Q_D$ or $Q_S$ 
  $$Q_D = 1600-300(0.2) = 1540$$ 
  $$Q_S = 1400+700(0.2) = 1540$$
\end{example}

\subsection{Comparative Statics (changes in equilibrium)} 

The following describes different cases of changes in equilibrium and some of the results$\dots$ 

\begin{itemize}
  \item Let's assume that the \emph{income of the consumers} (M) \emph{increases} and that we are analyzing a \emph{normal} good. We can, thus, conclude that the demand curve is going to \emph{shift to the right} (in the positive direction). The supply curve will remain unchanged, but there is a movement along the supply curve. The equilibrium point will increase in both price and quanity. 
  \item Let's assume that the \emph{cost of production} \emph{decreases}. We see that the supply curve will shift to the right. There is also a movement along (down) the demand curve. The equilibrium will decrease in price and increase in quantity. 
  \item Let's assume we are analyzing a normal good and that the \emph{cost of production decreases}. The demand curve shifts to the right. The supply curve shifts to the right as well. The equilibrium will have an increase in quantity, but there is price change guaranteed; the price could increase, remain the same, or even decrease. 
\end{itemize}

\subsection{Applications of Demand-Supply Model}

\begin{enumerate}
  \item \define{Price ceiling} is a maximum price that sellers may charge for a good/service which is set by the government. \\ 

    \bold{Example}: the market for gasoline in 1974. Due to supply issues (embargos), the government issues a price ceiling that prohibited increasing gasoline prices above a certain dollar amount. The price ceiling was set below the equilibrium price and this caused a significant shortage of gasoline. \\ 

    \bold{Example}: rent controls which limit how much landlords can charge in monthly rent and how much they can increase rent. 

  \item \define{Price floor} is a minimum price below which exchange is not permitted. \\

    \bold{Example}: the labor market. Imposing a minmum wage will increase the number of laborers willing to work but decrease firms demand for laborers due to increased price. This causes a surplus of labor which is called \emph{unemployment}. 
\end{enumerate}

A \define{price ceiling} has to be placed below the equilibrium point or else the price will just fall back into equilibrium. The opposite is true for \define{price floor} in that it must be above the equilibrium point or else the price would just settle back into equilibrium. 

\section{Introduction to Macroeconomics}

\define{Microeconomics} examines the functioning of individual industries and the behavior of individual decision-making units---that is, firms and households. \define{Macroeconomics} focuses on the economic behavior of aggregates---income, employment, output, and so on---on a national scale. Macroeconomics studies$\dots$ 
 \begin{enumerate}
   \item national income, not household income 
   \item the overall price level, not individual proces 
   \item total employment in the economy, not the demand for labor in specific markets
 \end{enumerate}

 \define{Aggregate behavior} is the behavior of all housholds and firms together. GDP is an aggregae. So is national income. Macroeconomics study aggregate consumption and aggregate investment intesively. \\ 

 The three major concerns of macroeconomics are output growth, unemployment, and inflation and deflation. 
 \begin{enumerate}
   \item \bold{Output growth}
     \begin{itemize}
       \item A \define{business cycle} is the cycle of short-term ups and downs in the economy.  
       \item \define{Aggregate output} is the total quantity of goods and services produced in an economy in a given period. Aggregate output is usually measured by \emph{gross domestic product} (GDP)
       \item A \define{business cycle} has four phases: 
         \begin{itemize}
           \item An \define{expansion} or boom is the period in the business cycle from a trough up to a peak during which output and employment grow 
           \item A \define{contraction}, \define{recesion}, or slump is the period in the business cycle from a peak down to a trough during which output and employment fall. A \define{depression} is prolonged and deep recession. 
           \item A \define{peak} is the transition from an expansion to a recession 
           \item A \define{trough} is the transition from a recession to an expansion 
         \end{itemize}
     \end{itemize}
   \item \bold{Unemployment}
     \begin{itemize}
       \item The \define{unemployment rate} is the ratio of the number of people unemployed to the total number of people in the labor force. People who are not acively seeking employment are not counted as unemployed because they are not in the labor force. 
       \item Unemployment above some minimum level implies the labor market is not in equilibrium 
     \end{itemize}
   \item \bold{Inflation} and \bold{Deflation}
     \begin{itemize}
       \item \define{Inflation} is an increase in the overall price level. \define{Hyperinflation} si a period of very rapid increases in the overall price level. A widely accepted definition of hyperinflation is inflation rates in excess of 50 percent per month. 
       \item \define{Deflation} is a decrease in the overall price level
     \end{itemize}
 \end{enumerate}

There are three \bold{market arenas}:
\begin{enumerate}
  \item \define{Goods-and-Services Markets}: households and the government purchase goods and services from firms 
  \item \define{Labor market}: interaction in the labor market takes place when firms and the government purchase labor from households 
  \item \define{Money market}: the money market is where households purchase stocks and bonds from firms. 
\end{enumerate}

\define{Fiscal policy} is the government's spending and taxing policies.  \\ 

\subsection{Gross Domestic Product (GDP)}

\define{Gross Domestic Product (GDP)} is the total market value of all final goods and services produced within a given period by factors or production located within a country. 
\begin{itemize}
  \item \define{Final goods and services} are goods and services produced for final use
  \item \define{Intermediate goods} are goods that are produced by one firm for use in further processing or for resale by another firm
  \item \define{Value added} is the difference between the value of goods as they leave a stage of production and the cost of the goods as they entered that stage
  \item \define{Gross national product (GNP)} is the total market value of all final goods and services produced within a given period by factors of production owned by a country's citizens, regardless of where the output is produced. 
\end{itemize}
\bold{GDP} is measured \emph{quarterly} in the United States. 

\begin{remark}
  Used products that are resold do not count towards a country's GDP. GDP only accounts for new goods that are sold to consumers.
\end{remark}

\subsubsection{Calculating GDP}

The \define{expenditure approach} is a method of computing GDP that measures the total amount spent on all final goods and services during a given period. The \define{income approach} is a method of computing GDP that measures the income---wages, rents, interest, and profits---received by all factors of production in producing final goods and services. We can say that $$GDP = C + I + G + (EX - IM)$$ where $C$ represents consumption, $I$ represents investments, $G$ represents government spending, $EX$ represents exports and $IM$ represents imports. 

\begin{definition}[Personal Consumption Expenditures (C)]
  expenditures by consumers on goods and services. There are three main categories:
  \begin{enumerate}
    \item \define{Durable goods} are goods that last a relatively long time, such as cars and household appliances 
    \item \define{Nondurable goods} are goods that are used up fairly quickly, such as food and clothing 
    \item \define{Services} are things we buy that do no involve the production of physical things, such as legal and medical services and education
  \end{enumerate}
\end{definition}

\begin{definition}[Gross Private Domestic Investment (I)]
  the total investment in capital---that is, the purchase of new housing, plants, equipment, and inventory by the private (or nongovernment) sector. \define{Investment} is the purchase of new capital---housing, plants, equipment, and inventory.  
  \begin{itemize}
    \item \define{Nonresidential investment} includes expenditures by firms for machines, tools, plants, and so on 
    \item \define{Residential investment} includes expenditures by households and firms on new houses and apartment buildings 
    \item The \define{change in business inventories} is the amount by which firms' inventories change during a period. inventories are the goods that firms produce now but intend to sell later 
  \end{itemize}
\end{definition}

\begin{definition}[Government Consumption and Gross Investment (G)]
  includes expenditures by federal, state, and local governments for final goods and services. It does not include government transfer payments or interest payments on the national debt because neither is a payment for any final goods or services. 
\end{definition}

\begin{definition}[Net Exports ($EX-IM$)]
  the difference between exports (sales to foreigners of US-produced goods and services) and imports (US-purchases of goods and services from abroad). The figure can be positive or negative. 
\end{definition}

\begin{definition}[Nominal GDP]
  GDP measured in \emph{current dollars}, the current prices we pay for goods and services. This is \bold{not} a \bold{desirable} measure of production; nominal GDP can increase because the price level has increased with no change in output.
\end{definition}

\begin{definition}[Real GDP]
  nominal GDP but adjusted for price: value GDP at constant prices
\end{definition}

\begin{remark}
  Nominal GDP is equal to the real GDP for the base year
\end{remark}

\begin{definition}[GDP Price Index]
  measures the current level of price relative to the base year. $$GDP_{\textrm{price index}} = \frac{\textrm{nominal GDP}}{\textrm{real GDP}} \times 100$$
\end{definition}

\subsection{Unemployment}

An \emph{employed} individual is any person 16 years or older (1) who works for pay, either for someone else or in his own business for 1 or more hours per week, (2), who works without pay for 15 or more hours per week in a family enterprise, or (3) who has a job but has been temporarily absent with or without pay. An \define{unemployed} individual is someone 16 years or older who is not working, is available for work, and has made specific efforts to find work during the pervious 4 weeks. Someone \define{not in the labor force} is any person who is not looking for work because he or she does not want a job or has given up looking. \\ 

The \define{labor force} is the number of people employed plus the number of unemployed. $$\textrm{labor force}=\textrm{employed}+\textrm{unemployed}$$ $$\textrm{population}=\textrm{labor force}+\textrm{not in labor force}$$ The \define{unemployment rate} is the ratio of the number of people employed to the total number of people in the labor force. $$\textrm{unemployment rate} = \frac{\textrm{employed}}{\textrm{employed} + \textrm{unemployed}}$$ The \define{labor-force participation rate} is the ratio of the labor force to the total population 16 years old or older. $$\textrm{labor force participation rate} = \frac{\textrm{labor force}}{\textrm{population}}$$

\begin{definition}[Types of Unemployment].
  \begin{enumerate}
    \item \define{Frictional unemployment} is the portion of unemployment that is due to the normal working of the labor market; used to denote short-run job/skill matching problems. Frictional unemployment is good for the economy because these workers will usually find a job that suits them better (and in which they are likely to be more productive). 
    \item \define{Structural unemployment} is the portion of unemployment that is a result of changes in the structure of the economy that results in a significant loss of jobs in certain industries. Structural unemployment creates longer-run adjustment problems that may last for years.  
    \item \define{Cyclical unemployment} is unemployment that is above frictional plus structural unemployment. Cyclical unemployment is the increase in unemployment that occurs during recessions and depressions. 
  \end{enumerate}
  The \define{natural rate of unemployment} is the unemployment rate that occurs as a normal part of the functioning of the economy. Sometimes taken as the sum of the frictional unemployment rate and the structural unemployment rate. 
\end{definition}
A \define{discouraged worker} is someone who is unemployed but recently stopped looking for works. The \define{discouraged-worker effect} is the decline in the measured unemployment rate that results when people who want to work but cannot find jobs grow discouraged and stop looking, thus dropping out of the ranks of the unemployed and the labor force. \\ 

\subsection{Inflation}

Not all price increases are inflation. Over any time period, prices of some goods will rise and other prices will fall. \define{Inflation} is an increase in the overall (average) price level. \define{Deflation} is a decrease in the overall (average) price level. Inflation happens when prices of many goods and services increase together. \\ 

The \define{consumer price index (CPI)} is a price index computed each month by the Bureau of Labor Statistics using a bundle that is meant to represent the "market basket" purchased monthly by the typical urban consumer. The CPI \emph{market basket} uses prices from about 71,000 goods and services from 22,000 outlets in 44 geographic areas. \define{Producer price indexes (PPI)} are measures of prices that producers receive for products at various stages in the production process. There are three main PPI categories:
\begin{itemize}
  \item finished goods 
  \item intermediate materials 
  \item crude materials
\end{itemize}

The Bureau of Labor Services uses the following steps to calculate inflation: 
\begin{enumerate}
  \item Fix the \emph{market basket} 
  \item Find the prices 
  \item Compute the cost of the basket for each year 
  \item Choose a base year and compute CPI 
  \item Compute inflation rate using CPI $$\textrm{Inflation rate} = \frac{\textrm{new} - \textrm{old}}{\textrm{old}} \times 100$$
\end{enumerate}

\begin{remark}
  The \bold{CPI} is always 100 for the base year (because you are dividing the CPI by itself $\times 100$)
\end{remark}

\section{Aggregate Expenditure and Equilibrium Output}

When aggregate output increases, additional income is generated and vice versa. The act of production creates income. 
\begin{itemize}
  \item \define{Aggregate output} is the total quantity of goods and services produced in an economy in a given period. 
  \item \define{Aggregate income} is the total income received by all factors of production in a given period. 
  \item \define{Aggregate output (income) (Y)} is a combined term used to remind you of the exact equality between aggregate output and aggregate income. 
\end{itemize} 

\subsection{The Market for Goods and Services (Keynesian Model)}

\subsubsection{Income and Spending}

Keynes noted that increases in income cause increases in consumption spending. Equally important, he noted that the increase in consumption would usually be less than the increase in income. \\ 

The \define{consumption function} is the relationship between consumption and income. The \define{marginal propensity to consume (MPC)} is the fraction of a change in income that is consumed, or spent. The MPC is also the slope of the consumption function $C = a+bY$ where $a$ is \emph{autonomous consumption} and $bY$ is the \emph{consumption related to income}. The MPC is the $b$ coefficient in a linear consumption function. 
\begin{align*}
  &\textrm{marginal propensity to consume} \\ 
  &= \textrm{slope of consumption function} \\ 
  &=MPC = \frac{\Delta C}{\Delta Y} = b
\end{align*}
\define{Aggregate savings} is the part of aggregate income that is not consumed 
$$S \equiv Y-C$$ 
This equation is an \emph{identity}, something that is always true by definition. The \define{marginal propensity to save (MPS)} is that fraction of a change in income that is saved. The MPS is also the slope of the saving function $S= ^{-}a^{+}(1-b)Y$.
$$MPS = \frac{\Delta S}{\Delta Y}$$
In this simple world, $MPC + MPS \equiv 1.00$. Households only consume and save. They do not invest. Investment is business spending on new plant and equipment.

\subsubsection{Investment}

\define{Investment} is purchases by firms of new buildings and equipment and additions to inventories. Spending on buildings and equipment is called \define{business fixed income}. Fixed investment is the gross increase in the capital stock. Fixed investment minus depreciation is thet net increase in the capital stock. \\ 

\subsubsection{Equilibrium Output}

\define{Equilibrium} is the condition that exists when quantity supplied and demanded are equal. At equilibrium, there is no tendnacy for price to change. \define{Aggregate Expenditure} (AE) is the total amount the economy plans to spedn in a given period (i.e. the spending by consumers combined with the spending by the firms) $$AE = C+I$$ 

\begin{definition}[Macroeconomic Equilbrium]
  when the aggregate output is equal to the aggregate spending. This occurs when the GDP is equal to the aggregate expenditure 
  \begin{align*}
    Y &= AE 
    Y &= C + I
  \end{align*}
\end{definition}

\define{Disequilibrium} means the economy is not in equilibrium. If output (income) is greater than planned aggregate expenditure, there will be an unplanned inventory increase. The signal to businesses is warehouses filling up. Businesses respond by reducing production, moving the economy toward equilibrium. If output (income) is less than planned aggregate expenditure, there will be an unplanned inventory decrease. The signal to businesses is warehouses emptying. Businesses repond by increasing producting, moving the economy toward equilbrium. 

\subsubsection{The Multiplier}

A change in planned investment will have a multiplied impact on equilibrium income. The \define{multiplier} is a number that indicates how much the equilibrium level of output changes as a result of a change in some autonomous variable.
  $$\textrm{Multiplier} = \frac{\Delta Y}{\Delta \textrm{autonomous variable}}$$
  $$\frac{\Delta Y}{\Delta I} = \frac{1}{MPS}$$
  $$\frac{1}{MPS} = \frac{1}{(1 - MPC)}$$
The increase in planned investment spending raises income paid to those working on the investment projects. They will increase their spending and saving. Higher spending creates more income, which is again spent. The size of the multiplier depends on the slope of the aggregate planned expenditure line. 

\subsection{The Government and Fiscal Policy}

There are two variables controlled by the government: \define{government spending} (G) on goods and services and \define{net taxes}. \bold{Net taxes} (T) consist of the \emph{collection of taxes} from individuals and firms minus the \emph{transfer payments} such as social security and unemployment benefits for example.  

\begin{definition}[Disposable/after-tax Income ($Y_D$)]
  the total income minus net taxes: $Y_D = Y-T$. Naturally, $Y-T \equiv C+S$ and $Y \equiv C+S+T$. 
\end{definition}

Let $G$ represent government spending and $T$ net taxes (government income). There are three different types of government budgets$\dots$ 
\begin{itemize}
  \item \define{Deficit}: $G-T > 0 \implies G > T$
  \item \define{Surplus}: $G-T < 0 \implies G < T$ 
  \item \define{Balanced budget}: $G-T = 0 \implies G=T$
\end{itemize}

Now including government spending, we update the definition of \define{aggregate spending} (AE) to $$AE = C + I + G$$ where $C$ is spending by consumers, $I$ is spending by firms, and $G$ is spending by the government. Still the macroeconomic equilbrium must equal aggregate spending (i.e. $Y = AE$). We update the consumption function to include the taxes that are paid so that $$C=a+bY_d$$ $$C=a+b(Y_T)$$ 
\begin{example}
  Let $I=100$, $G=100$, $T=100$, and the consumption function $C=10+0.75Y_d$. Calculate the equilbrium Y. 
  \begin{gather}
    Y=AE \\ 
    Y=C+I+G \\ 
    Y = (100 + 0.75Y_d) + 100 + 100 \\ 
    Y = 300 + 0.75Y_d \\ 
    Y = 300 + 0.75(Y-100) \\ 
    Y = 300 + 0.75Y - 75 \\ 
    Y - 0.75Y = 225 \\ 
    0.25Y = 225 \\
    Y = 900
  \end{gather}
\end{example}

\begin{definition}[Fiscal policy]
  refers to the government's spending and taxing policies. Fiscal policy includes changes in government purchases of goods and services (mainly labor services), taxes, and/or transfer payments to households with the objective of changing the economy's growth. The two types of \define{fiscal policy} are: 
  \begin{itemize}
    \item \define{Expansionary} fiscal policy is when government spending increases and/or taxation decreases (GDP increases). 
    \item \define{Contractionary} fiscal policy is when government spending decreases and/or taxation increases (GDP decreases). 
  \end{itemize}
\end{definition}

The \define{government spending multiplier} is a number that indicates how much the equilbrium level of output changes as a result of a change in government spending. Increased spending will lead to increased output, which leads to increased income. More workers are employed, and they in turn act as consumers and spend their incomes. $$\textrm{government spending multiplier} = \frac{\Delta Y}{\Delta G} \frac{1}{MPS} = \frac{1}{1-MPC}$$

The \define{tax multiplier} is a number that indicats the change in the equilibrium level of output when there is a change in taxes. To reduce unemployment without increasing government spending, taxes must be cut. This increases disposable income, which is likely to add to consumption, which will lead to an increase in output and employment, and hence income. $$\textrm{tax multiplier} = \frac{\Delta Y}{\Delta T} = - \frac{MPC}{MPS} = - \frac{MPC}{1-MPC}$$

\begin{example}[Effect of a change in government spending vs. taxes]
  Let's assume that government spending ($G$) increases by 200. How does this affect $Y$?
  $$\frac{\Delta Y}{\Delta G} = \frac{1}{MPS} = \frac{1}{0.25} = 4 \rightarrow \Delta y = 4 \times \Delta G$$ 
  $$\Delta y = 4 \times 200 = 800$$
  $$y_{\textrm{new}} = y_{\textrm{old}} + 800 = 900 + 800 = 1700.$$
  Now, let's assume that there is a tax cut of 100. 
  $$\frac{\Delta Y}{\Delta T} = - \frac{MPC}{MPS} = \frac{0.75}{0.25} = -3$$
  $$\Delta y = (-3) \Delta T = (-3)(-100) = 300$$
  $$y_{\textrm{new}} = y_{\textrm{old}} + 300 = 900 + 300 = 1200.$$
\end{example}

\begin{example}
  What happens to the GDP (Y) if taxes and spending increase by the same amount? \\ 

  Suppose the government increases spending by 40 and to finance this spending it increases taxes by 40 (balanced budget).  
  $$\Delta y = 4 \times \Delta G = 4 \times 40 = 160 \implies y \textrm{ increases by } 160$$
  $$\Delta y = (-3) \Delta T = (-3)(40) \implies y \textrm{ decreases by } 120$$
\end{example}

\begin{definition}[Balanced Budget Multiplier]
  a number that indicates how much the equilbrium $y$ changes as a result to an \bold{equal} change in government spending ($G$) and taxation ($T$) 
  $$\textrm{If } \Delta G = \Delta T \implies \Delta y = \Delta G = \Delta T$$
\end{definition}

\section{The Money Market}

The money market is controlled by the \define{Federal Reserve Bank} (Fed). The Federal Reserve Bank measures the money supply in the US with \define{transactions money} (M1) and \define{broad money} (M2). 

\begin{definition}[Transactions Money]
  money that can be directly used for transactions. 
  \begin{align*}
    M1 = &\textrm{ money currently held outside of the banks} \\ 
    + &\textrm{ demand deposits} \\ 
    + &\textrm{ traveler's checks} \\ 
    + &\textrm{ other checkable deposits}
  \end{align*}
\end{definition}

\begin{definition}[Broad Money]
  money that can be directly used for transactions plus other savings
  \begin{align*}
    M2 = &\textrm{ M2 } \\ 
    + &\textrm{ savings account} \\
    + &\textrm{ money market accounts} \\ 
    + &\textrm{ other near monies} \\ 
  \end{align*}
\end{definition}

\end{document}
