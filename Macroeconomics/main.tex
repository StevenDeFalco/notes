% A note-taking template by Steven DeFalco
% github.com/StevenDeFalco/notes

\documentclass{article}

% import note styles
\usepackage{../styles}

% Heading information
\title{BT243: Macroeconomics Notes}
\author{Steven DeFalco}
\date{Fall 2023}


\begin{document}


\maketitle
\tableofcontents
\newpage


% Notes start here

\section{Introduction}

\subsection{Basic Definitions}

\define{Scarcity} is a situation in which resources are limited and can be used in different ways, so we \emph{must sacrifice one thing for another}. \\ 

\define{Labor} is the primary source for providing goods and services. \\

\define{Opportunity cost} is what we give up when we make a choice or a decision. \\

\define{Economics} is the study of the choices made by people (\emph{individuals and societies}) who are faced with scarcity. The two major fields are microecnomics and macroeconomics. \emph{Microecnomics} studiees consumers and produces. \emph{Macroeconomics} studies the economy as a whole. \\ 

\subsubsection{Economic questions a society is faced with...}

\begin{enumerate}
  \item What will be produced? 
  \item How will it be produced?
  \item Who consumes the goods and services produced?
\end{enumerate}

\subsection{Production Possibility Frontier (PPF)}

The \define{PPF} is a graph that shows all possible combinations of goods and services that can be produced if all resources used efficiently. This is a limit that cannot be exceeded; it represents the best case scenario (in terms of efficiency). 

\begin{example}
Consider two goods: defense goods and non-defense goods. With limited resources only certain combinations can be produced; these include the following:
  \begin{itemize}
    \item[A] 200 units of defense goods and 0 units of non-defense goods
    \item[B] 195 units of defense goods and 25 units of non-defense goods (opp. cost = 5)
    \item[C] 188 units of defense goods and 50 units of non-defense goods (opp. cost = 7) 
    \item[D] 175 units of defense goods and 75 units of non-defense goods (opp. cost = 13)
    \item[E] 155 units of defense goods and 100 units of non-defense goods (opp. cost = 20)
    \item[F] 125 units of defense goods and 125 units of non-defense goods (opp. cost = 30)
    \item[G] 75 units of defense goods and 150 units of non-defense goods (opp.cost = 50)
    \item[H] 0 units of defense goods and 160 units of non-defense goods (opp. cost = 75)
  \end{itemize}
\emph{Opportunity cost increases} as specialization in inputs to the labor must be given up. 
\end{example}

Graphically, the \define{PPF} represents the barrier between inefficient use of resources and unachievable levels of prdouction given the scenario. A coordinate under the PPF curve indicates under-utilization of resources A coordinate above the PPF is an impossible combination and, by definiton, unachievable. \\

The PPF can be shifted (in the positive direction) if there is an \emph{increase in resurces} or an \emph{improvement in technology}; this is called \define{economic growth}. \define{International trade} can help a nation move beyond their maximum capabilities in terms of consumption (exceed their PPF). 

\section{Demand and Supply Model}

\define{Firms} produce and supply their output to the \define{consumers} who demand the product. 

\subsection{Demand}

The quantity (Q) \define{demanded} is the amount of a good or service that consumers are willing and able to buy. Willingness and ability are the primary determinants of demand. Price of a product is the main determinant of our willingness and ability to purchase a product. \\ 

\bold{Determinants} of demand: 
\begin{enumerate}
  \item Price of the product (P) \\ 
    \begin{itemize}
      \item As the price of a product increases, the quantity of the product demanded decreases. \\ 
      \item As the price of a product decreases, the quantity of the product demanded increases. \\ 
      \item The \define{Law of Demand} is the negative relationship between price and quantity. 
    \end{itemize}
  \item Income (M)
    \begin{itemize}
      \item As income increases, the quantity of products demanded (typcially) increases. 
      \item As income decreases, the quantity of products demanded decreases. 
      \item For some products these relationships are the opposite. These products are considered \define{inferior goods}; such products are demanded more when income is lower due to the nature of the product (e.g. fast food or bus rides). 
    \end{itemize}
  \item Prices of related goods 
    \begin{itemize}
      \item When an increase in the price of one good causes the demand for another good to increase, the two goods are called \define{substitutes}. 
      \item For example eating at restaurants and at-home can be considered \emph{substitutes}. If the price of eating at restaurants increases and causes a greater demand for eating-at-home/grocery-shopping (for example), then these two products are considered \bold{substitutes}. 
      \item When an increase in the price of one good causes the demand for another good to decrease, the two goods are called \define{complements}. 
      \item For example gasoline and big-cars can be considered \emph{complements} because as the price of gasoline increases, the demand for big-cars (that burn a lot of gas!) decreases. 
    \end{itemize}
  \item Taste and preferences of consumers 
  \item Expectations of consumers 
    \begin{itemize}
      \item This refers to consumers' beliefs about future income and prices.
    \end{itemize}
\end{enumerate} 

\begin{example}
  Ice-cream cones price-quantity relationship
  \begin{center}
    \begin{tabular}{ | c | c | }
      \hline
      \bold{P(\$)} & \bold{Q} \\
      \hline
      0 & 12 \\ 
      0.5 & 10 \\ 
      1 & 8 \\ 
      1.5 & 6 \\ 
      2 & 4 \\ 
      2.5 & 2 \\ 
      3 & 0 \\ 
      \hline 
    \end{tabular}
  \end{center}

  There is a \emph{negative relationship} between P and Q. The graph of price vs. quantity is the \bold{demand curve}. The \emph{demand curve} is linear. If income changes, the original relationship between price and quantity changes; thus, the demand curve will shift.

\end{example}

\begin{remark}
  A change in price results in movement along the demand curve. 
\end{remark}

\begin{remark}
  Demand curve shifting...
  \begin{itemize}
    \item The demand curve shifts to the right when there is an increase in demand. 
    \item The demand curve shifts to the left when there is a decrease in demand.  
  \end{itemize}
\end{remark}

\begin{definition}
  \define{celeris paribus} means "other things being equal". For example when modeling examples, we draw conclusions given that everything besides what we specifically study is constant. 
\end{definition}

\begin{definition}[Market demand]
  the sum of all individual demands. 
\end{definition}

\subsection{Supply}

The \define{quantity supplied} is the amount of a good or service that sellers are willing and able to sell. Some determinants of the \emph{quantity supplied} are as follows$\dots$

\begin{enumerate}
  \item \bold{Price (P)}: as the price (of the product/service) increases, the quantity supplied increases; as the price decreases, the quantity supplied decreases. This is called the \define{law of supply}. 
  \item \bold{Cost of production}: as the cost of production increases, the quantity supplied decreases; as the cost of production decreases, the quantity supplied increases. 
  \item \bold{Number of producers}
  \item \bold{Expectations of the firms}: beliefs about future performance
\end{enumerate}

The \define{supply curve} shows the positive relationship between price and quantity (P/Q). A change in the price results in movement along the supply curve. If any of the other determinants change then there is a shift of the supply curve. \\ 

\begin{remark}
  An increase in quantity supplied results in a \emph{right-shift} of the supply curve. A decrease in the quantity supplied results in a \emph{left-shift} of the supply curve. 
\end{remark}

\begin{definition}[Market supply]
  the sum of all that is supplied by all producers.
\end{definition}

\begin{definition}[Market equilibrium]
  a situation in which the quantity demanded is equal to the quantity supplied. $$Q_{D}=Q_{S}$$ The market equilibrium is---graphically---where the supply and demand curves intersect: when the quantity demanded and supplied are exactly the same. 
\end{definition}

Markets that are \emph{not} in equilibrium are \bold{inefficient}. 
\begin{itemize}
  \item \define{Excess supply (surplus)} refers to a scenario where producers are willing to produce more than consumers are willing to buy. In this case, the price will start to decrease and the quantity supplied decreases/quantity demanded increases until equilibrium is reached. 
  \item \define{Excess demand (shortage)} refers to a scenario where the consumers are willing to buy more than producers are willing to produce. The price will start to increase and---thus---the quantity demanded will decrease and the quantity produced will increase until equilibrium is met. 
\end{itemize}

\begin{example}{Solving for equilibrium}
  \\ Demand curve: $Q_D = 1600 - 300P$
  \\ Supply curve: $Q_S = 1400 + 700P$  \\ 
  Equilibrium: $Q_D = Q_S$ 
  $$1600-300P=1400+700P$$
  $$1000P=200$$
  $$P=0.2$$
  Substitute $P$ into either $Q_D$ or $Q_S$ 
  $$Q_D = 1600-300(0.2) = 1540$$ 
  $$Q_S = 1400+700(0.2) = 1540$$
\end{example}

\subsection{Comparative Statics (changes in equilibrium)} 

The following describes different cases of changes in equilibrium and some of the results$\dots$ 

\begin{itemize}
  \item Let's assume that the \emph{income of the consumers} (M) \emph{increases} and that we are analyzing a \emph{normal} good. We can, thus, conclude that the demand curve is going to \emph{shift to the right} (in the positive direction). The supply curve will remain unchanged, but there is a movement along the supply curve. The equilibrium point will increase in both price and quanity. 
  \item Let's assume that the \emph{cost of production} \emph{decreases}. We see that the supply curve will shift to the right. There is also a movement along (down) the demand curve. The equilibrium will decrease in price and increase in quantity. 
  \item Let's assume we are analyzing a normal good and that the \emph{cost of production decreases}. The demand curve shifts to the right. The supply curve shifts to the right as well. The equilibrium will have an increase in quantity, but there is price change guaranteed; the price could increase, remain the same, or even decrease. 
\end{itemize}

\subsection{Applications of Demand-Supply Model}

\begin{enumerate}
  \item \define{Price ceiling} is a maximum price that sellers may charge for a good/service which is set by the government. \\ 

    \bold{Example}: the market for gasoline in 1974. Due to supply issues (embargos), the government issues a price ceiling that prohibited increasing gasoline prices above a certain dollar amount. The price ceiling was set below the equilibrium price and this caused a significant shortage of gasoline. \\ 

    \bold{Example}: rent controls which limit how much landlords can charge in monthly rent and how much they can increase rent. 

  \item \define{Price floor} is a minimum price below which exchange is not permitted. \\

    \bold{Example}: the labor market. Imposing a minmum wage will increase the number of laborers willing to work but decrease firms demand for laborers due to increased price. This causes a surplus of labor which is called \emph{unemployment}. 
\end{enumerate}

A \define{price ceiling} has to be placed below the equilibrium point or else the price will just fall back into equilibrium. The opposite is true for \define{price floor} in that it must be above the equilibrium point or else the price would just settle back into equilibrium. 

\section{Introduction to Macroeconomics}

\define{Microeconomics} examines the functioning of individual industries and the behavior of individual decision-making units---that is, firms and households. \define{Macroeconomics} focuses on the ecnomic behavior of aggregates---income, employment, output, and so on---on a national scale. Macroeconomics studies$\dots$ 
 \begin{enumerate}
   \item national income, not household income 
   \item the overall price level, not individual proces 
   \item total employment in the economy, not the demand for labor in specific markets
 \end{enumerate}

 \define{Aggregate behavior} is the behavior of all housholds and firms together. GDP is an aggregae. So is national income. Macroeconomics study aggregate consumption and aggregate investment intesively. \\ 

 The three major concerns of macroeconomics are output growth, unemployment, and inflation and deflation. 
 \begin{enumerate}
   \item \bold{Output growth}
     \begin{itemize}
       \item A \define{business cycle} is the cycle of short-term ups and downs in the economy.  
       \item \define{Aggregate output} is the total quantity of goods and services produced in an economy in a given period. Aggregate output is usually measured by \emph{gross domestic product} (GDP)
       \item A \define{business cycle} has four phases: 
         \begin{itemize}
           \item An \define{expansion} or boom is the period in the business cycle from a trough up to a peak during which output and employment grow 
           \item A \define{contraction}, \define{recesion}, or slump is the period in the business cycle from a peak down to a trough during which output and employment fall. A \define{depression} is prolonged and deep recession. 
           \item A \define{peak} is the transition from an expansion to a recession 
           \item A \define{trough} is the transition from a recession to an expansion 
         \end{itemize}
     \end{itemize}
   \item \bold{Unemployment}
     \begin{itemize}
       \item The \define{unemployment rate} is the ratio fo the number of people unemployed to the total number of people in the labor force. People who are not acively seeking employment are not counted as unemployed because they are not in the labor force. 
       \item Unemployment above some minimum level implies the labor market is not in equilibrium 
     \end{itemize}
   \item \bold{Inflation} and \bold{Deflation}
     \begin{itemize}
       \item \define{Inflation} is an increase in the overall price level. \define{Hyperinflation} si a period of very rapid increases in the overall price level. A widely accepted definition of hyperinflation is inflation rates in excess of 50 percent per month. 
       \item \define{Deflation} is a decrease in the overall price level
     \end{itemize}
 \end{enumerate}

There are three \bold{market arenas}:
\begin{enumerate}
  \item \define{Goods-and-Services Markets}: households and the government purchase goods and services from firms 
  \item \define{Labor market}: interaction in the labor market takes place when firms and the government purchase labor from households 
  \item \define{Money market}: the money market is where households purchase stocks and bonds from firms. 
\end{enumerate}

\define{Fiscal policy} is the government's spending and taxing policies.  \\ 

\define{Gross Domestic Product (GDP)} is the most basic measure of how an economy is performing. GDP is the total market value of all final goods and services produced within a given period by factors or production located within a country. \define{Final goods and services} are goods and services produced for final use. While \define{intermediate goods} are goods that are produced by one firm for use in further processing or for resale by another firm. \define{Value added} is the difference between the value of goods as they leave a stage of production and the cost of the goods as they entered that stage. \define{Gross national product (GNP)} is the total market value of all final goods and services produced within a given period by factors of production owned by a country's citizens, regardless of where the output is produced. 

\end{document}
